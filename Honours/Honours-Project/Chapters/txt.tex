T HE U NIVERSITY OF A UCKLAND
H ONOURS P ROJECT

Survey statistics in a database

Author:
Charco H UI

Supervisor:
Professor Thomas L UMLEY

A thesis submitted in fulfilment of the requirements
for the degree of Bachelor of Science (Honours)
in the
Department of Statistics

May 23, 2018

iii

THE UNIVERSITY OF AUCKLAND

Abstract
Department of Statistics
Bachelor of Science (Honours)
Survey statistics in a database
by Charco H UI
Multistage surveys can give rise to moderately large data sets (tens of millions of
rows). Most current software for survey analysis reads the data into memory, the
survey package in R provides fairly comprehensive analysis features for complex
surveys which are small enough to fit into memory easily, however, most of the
computations can actually be expressed as database operations. There is already a
similar approach with the sqlsurvey package in R which performs substantial computation in SQL in the database, importing only small summary tables into R, this
approach scales to very large surveys such as the American Community Survey and
the Nationwide Emergency Department Sample, but this approach causes compatible issues with different types of databases. Therefore, in this project I will work on
implementing R functions and testing some survey computations using the dplyr
and dbplyr R package as a database interface.

v

Acknowledgements
I would like to acknowledge my supervisor, Professor Thomas Lumley with my
deepest appreciation. I would like to thank him for his patience in sharing his expertise, without his help, this project would not be possible.
Lastly, i would like to thank my friends and family for their continuous support.

vii

Contents
Abstract

iii

Acknowledgements

v

1

.
.
.
.
.
.
.
.

1
1
1
2
2
2
3
3
4

.
.
.
.
.
.
.
.
.
.
.
.
.
.
.
.
.
.
.
.
.
.
.
.
.
.
.
.

5
5
7
7
7
8
8
9
9
9
10
10
11
11
11
12
13
14
15
15
15
15
16
16
16
16
18
18
18

2

Introduction
1.1 Background . . . . . . . . . . . . . . . . . . . . . . . . . . . . .
1.2 Survey data in SQL . . . . . . . . . . . . . . . . . . . . . . . . .
1.3 Coding with dplyr and dbplyr . . . . . . . . . . . . . . . . . .
1.3.1 Introduction to dplyr . . . . . . . . . . . . . . . . . . . .
1.3.2 Pipes . . . . . . . . . . . . . . . . . . . . . . . . . . . . .
1.3.3 dplyr’s SQL compatibility (dbplyr) . . . . . . . . . . . .
1.3.4 Quasi-quotation . . . . . . . . . . . . . . . . . . . . . . .
1.3.5 Common issues while coding with dplyr in a database
Methodology
2.1 Survey Design . . . . . . .
2.2 Population Total . . . . . .
2.2.1 Usage . . . . . . . .
2.2.2 Arguments . . . .
2.2.3 Examples . . . . .
2.2.4 Difficulties . . . . .
2.3 Population Mean . . . . .
2.3.1 Usage . . . . . . . .
2.3.2 Arguments . . . .
2.3.3 Examples . . . . .
2.3.4 Difficulties . . . . .
2.4 Regression . . . . . . . . .
2.4.1 Usage . . . . . . . .
2.4.2 Arguments . . . .
2.4.3 Examples . . . . .
2.4.4 Difficulties . . . . .
2.5 Quantiles . . . . . . . . . .
2.5.1 Usage . . . . . . . .
2.5.2 Arguments . . . .
2.5.3 Examples . . . . .
2.5.4 Difficulties . . . . .
2.6 Survey Tables . . . . . . .
2.6.1 Usage . . . . . . . .
2.6.2 Arguments . . . .
2.6.3 Examples . . . . .
2.7 Survey Statistic on Subsets
2.7.1 Usage . . . . . . . .
2.7.2 Arguments . . . .

.
.
.
.
.
.
.
.
.
.
.
.
.
.
.
.
.
.
.
.
.
.
.
.
.
.
.
.

.
.
.
.
.
.
.
.
.
.
.
.
.
.
.
.
.
.
.
.
.
.
.
.
.
.
.
.

.
.
.
.
.
.
.
.
.
.
.
.
.
.
.
.
.
.
.
.
.
.
.
.
.
.
.
.

.
.
.
.
.
.
.
.
.
.
.
.
.
.
.
.
.
.
.
.
.
.
.
.
.
.
.
.

.
.
.
.
.
.
.
.
.
.
.
.
.
.
.
.
.
.
.
.
.
.
.
.
.
.
.
.

.
.
.
.
.
.
.
.
.
.
.
.
.
.
.
.
.
.
.
.
.
.
.
.
.
.
.
.

.
.
.
.
.
.
.
.
.
.
.
.
.
.
.
.
.
.
.
.
.
.
.
.
.
.
.
.

.
.
.
.
.
.
.
.
.
.
.
.
.
.
.
.
.
.
.
.
.
.
.
.
.
.
.
.

.
.
.
.
.
.
.
.
.
.
.
.
.
.
.
.
.
.
.
.
.
.
.
.
.
.
.
.

.
.
.
.
.
.
.
.
.
.
.
.
.
.
.
.
.
.
.
.
.
.
.
.
.
.
.
.

.
.
.
.
.
.
.
.
.
.
.
.
.
.
.
.
.
.
.
.
.
.
.
.
.
.
.
.

.
.
.
.
.
.
.
.
.
.
.
.
.
.
.
.
.
.
.
.
.
.
.
.
.
.
.
.

.
.
.
.
.
.
.
.
.
.
.
.
.
.
.
.
.
.
.
.
.
.
.
.
.
.
.
.

.
.
.
.
.
.
.
.
.
.
.
.
.
.
.
.
.
.
.
.
.
.
.
.
.
.
.
.

.
.
.
.
.
.
.
.
.
.
.
.
.
.
.
.
.
.
.
.
.
.
.
.
.
.
.
.

.
.
.
.
.
.
.
.
.
.
.
.
.
.
.
.
.
.
.
.
.
.
.
.
.
.
.
.

.
.
.
.
.
.
.
.
.
.
.
.
.
.
.
.
.
.
.
.
.
.
.
.
.
.
.
.

.
.
.
.
.
.
.
.
.
.
.
.
.
.
.
.
.
.
.
.
.
.
.
.
.
.
.
.

.
.
.
.
.
.
.
.
.
.
.
.
.
.
.
.
.
.
.
.
.
.
.
.
.
.
.
.

.
.
.
.
.
.
.
.
.
.
.
.
.
.
.
.
.
.
.
.
.
.
.
.
.
.
.
.

.
.
.
.
.
.
.
.
.
.
.
.
.
.
.
.
.
.
.
.
.
.
.
.
.
.
.
.

.
.
.
.
.
.
.
.
.
.
.
.
.
.
.
.
.
.
.
.
.
.
.
.
.
.
.
.
.
.
.
.
.
.
.
.

.
.
.
.
.
.
.
.
.
.
.
.
.
.
.
.
.
.
.
.
.
.
.
.
.
.
.
.
.
.
.
.
.
.
.
.

.
.
.
.
.
.
.
.
.
.
.
.
.
.
.
.
.
.
.
.
.
.
.
.
.
.
.
.
.
.
.
.
.
.
.
.

.
.
.
.
.
.
.
.
.
.
.
.
.
.
.
.
.
.
.
.
.
.
.
.
.
.
.
.
.
.
.
.
.
.
.
.

viii

2.8

3

4

5

2.7.3 Examples . . . . . . . . . . . . . . . . . .
Replicate Weights . . . . . . . . . . . . . . . . . .
2.8.1 Replicate Standard Errors . . . . . . . . .
2.8.2 Survey Replicate Design . . . . . . . . . .
2.8.3 Examples . . . . . . . . . . . . . . . . . .
2.8.4 Population Total with replicate weights .
2.8.5 Population Mean with replicate weights .

Graphics
3.1 Histogram . . . . . . . . . .
3.1.1 Usage . . . . . . . . .
3.1.2 Arguments . . . . .
3.1.3 Examples . . . . . .
3.1.4 Difficulty . . . . . . .
3.2 Box Plot . . . . . . . . . . . .
3.2.1 Usage . . . . . . . . .
3.2.2 Arguments . . . . .
3.2.3 Examples . . . . . .
3.3 Hexagon Binning . . . . . .
3.3.1 Usage/Arguments .
3.3.2 Examples . . . . . .
3.3.3 Difficulties . . . . . .
3.4 Conditional Plots . . . . . .
3.4.1 Function description
3.4.2 Examples . . . . . .
3.5 Why ggplot? . . . . . . . . .

.
.
.
.
.
.
.

.
.
.
.
.
.
.

.
.
.
.
.
.
.

.
.
.
.
.
.
.

.
.
.
.
.
.
.

.
.
.
.
.
.
.

.
.
.
.
.
.
.

.
.
.
.
.
.
.

.
.
.
.
.
.
.

.
.
.
.
.
.
.

.
.
.
.
.
.
.

.
.
.
.
.
.
.

.
.
.
.
.
.
.

18
19
19
19
20
20
20

.
.
.
.
.
.
.
.
.
.
.
.
.
.
.
.
.

.
.
.
.
.
.
.
.
.
.
.
.
.
.
.
.
.

.
.
.
.
.
.
.
.
.
.
.
.
.
.
.
.
.

.
.
.
.
.
.
.
.
.
.
.
.
.
.
.
.
.

.
.
.
.
.
.
.
.
.
.
.
.
.
.
.
.
.

.
.
.
.
.
.
.
.
.
.
.
.
.
.
.
.
.

.
.
.
.
.
.
.
.
.
.
.
.
.
.
.
.
.

.
.
.
.
.
.
.
.
.
.
.
.
.
.
.
.
.

.
.
.
.
.
.
.
.
.
.
.
.
.
.
.
.
.

.
.
.
.
.
.
.
.
.
.
.
.
.
.
.
.
.

.
.
.
.
.
.
.
.
.
.
.
.
.
.
.
.
.

.
.
.
.
.
.
.
.
.
.
.
.
.
.
.
.
.

.
.
.
.
.
.
.
.
.
.
.
.
.
.
.
.
.

.
.
.
.
.
.
.
.
.
.
.
.
.
.
.
.
.

.
.
.
.
.
.
.
.
.
.
.
.
.
.
.
.
.

.
.
.
.
.
.
.
.
.
.
.
.
.
.
.
.
.

.
.
.
.
.
.
.
.
.
.
.
.
.
.
.
.
.

.
.
.
.
.
.
.
.
.
.
.
.
.
.
.
.
.

.
.
.
.
.
.
.
.
.
.
.
.
.
.
.
.
.

21
22
22
22
22
23
24
24
24
24
25
25
26
26
28
28
28
29

Results
4.1 Total . . . . . . . . . . . . . . . . . . . . .
4.1.1 Numeric variable . . . . . . . . .
4.1.2 Categorical variable with 7 levels
4.2 Mean . . . . . . . . . . . . . . . . . . . .
4.2.1 Numeric variable . . . . . . . . .
4.2.2 Categorical variable with 7 levels
4.3 Regression . . . . . . . . . . . . . . . . .
4.4 Quantiles . . . . . . . . . . . . . . . . . .
4.5 Survey Tables . . . . . . . . . . . . . . .
4.6 Total - Replicate Weights . . . . . . . . .
4.7 Mean - Replicate Weights . . . . . . . .
4.8 Histogram . . . . . . . . . . . . . . . . .
4.9 Boxplot . . . . . . . . . . . . . . . . . . .
4.10 Hexagon Binning . . . . . . . . . . . . .
4.11 Time Comparison . . . . . . . . . . . . .

.
.
.
.
.
.
.
.
.
.
.
.
.
.
.

.
.
.
.
.
.
.
.
.
.
.
.
.
.
.

.
.
.
.
.
.
.
.
.
.
.
.
.
.
.

.
.
.
.
.
.
.
.
.
.
.
.
.
.
.

.
.
.
.
.
.
.
.
.
.
.
.
.
.
.

.
.
.
.
.
.
.
.
.
.
.
.
.
.
.

.
.
.
.
.
.
.
.
.
.
.
.
.
.
.

.
.
.
.
.
.
.
.
.
.
.
.
.
.
.

.
.
.
.
.
.
.
.
.
.
.
.
.
.
.

.
.
.
.
.
.
.
.
.
.
.
.
.
.
.

.
.
.
.
.
.
.
.
.
.
.
.
.
.
.

.
.
.
.
.
.
.
.
.
.
.
.
.
.
.

.
.
.
.
.
.
.
.
.
.
.
.
.
.
.

.
.
.
.
.
.
.
.
.
.
.
.
.
.
.

.
.
.
.
.
.
.
.
.
.
.
.
.
.
.

.
.
.
.
.
.
.
.
.
.
.
.
.
.
.

.
.
.
.
.
.
.
.
.
.
.
.
.
.
.

.
.
.
.
.
.
.
.
.
.
.
.
.
.
.

31
32
32
32
33
33
33
34
34
35
35
36
36
37
37
38

.
.
.
.
.
.
.
.
.
.
.
.
.
.
.
.
.

.
.
.
.
.
.
.
.
.
.
.
.
.
.
.
.
.

.
.
.
.
.
.
.
.
.
.
.
.
.
.
.
.
.

.
.
.
.
.
.
.
.
.
.
.
.
.
.
.
.
.

.
.
.
.
.
.
.
.
.
.
.
.
.
.
.
.
.

.
.
.
.
.
.
.
.
.
.
.
.
.
.
.
.
.

Usability
39
5.1 The package . . . . . . . . . . . . . . . . . . . . . . . . . . . . . . . . . . 39
5.2 Supporting functions . . . . . . . . . . . . . . . . . . . . . . . . . . . . . 39
5.3 Testing . . . . . . . . . . . . . . . . . . . . . . . . . . . . . . . . . . . . . 39

ix
6

Discussion
6.1 Overview . . . . . . . .
6.2 Future work . . . . . .
6.2.1 Error Messages
6.2.2 Functions . . .

.
.
.
.

.
.
.
.

.
.
.
.

A Codes
A.1 svydbdesign . . . . . . . . .
A.2 svydbtotal . . . . . . . . . .
A.3 svydbmean . . . . . . . . . .
A.4 svydblm . . . . . . . . . . .
A.5 svydbquantile . . . . . . . .
A.6 svydbtable . . . . . . . . . .
A.7 svydbby . . . . . . . . . . .
A.8 svydbrepdesign . . . . . . .
A.9 svydbreptotal . . . . . . . .
A.10 svydbrepmean . . . . . . . .
A.11 svydbhist . . . . . . . . . . .
A.12 svydbboxplot . . . . . . . .
A.13 svydbhexbin, svydbhexplot
A.14 svydbcoplot . . . . . . . . .
A.15 Other Functions . . . . . . .

.
.
.
.

.
.
.
.

.
.
.
.

.
.
.
.

.
.
.
.

.
.
.
.

.
.
.
.

.
.
.
.

.
.
.
.

.
.
.
.

.
.
.
.

.
.
.
.

.
.
.
.

.
.
.
.

.
.
.
.

.
.
.
.

.
.
.
.

.
.
.
.

.
.
.
.

.
.
.
.

.
.
.
.

.
.
.
.

.
.
.
.

.
.
.
.

.
.
.
.

41
41
41
41
41

.
.
.
.
.
.
.
.
.
.
.
.
.
.
.

.
.
.
.
.
.
.
.
.
.
.
.
.
.
.

.
.
.
.
.
.
.
.
.
.
.
.
.
.
.

.
.
.
.
.
.
.
.
.
.
.
.
.
.
.

.
.
.
.
.
.
.
.
.
.
.
.
.
.
.

.
.
.
.
.
.
.
.
.
.
.
.
.
.
.

.
.
.
.
.
.
.
.
.
.
.
.
.
.
.

.
.
.
.
.
.
.
.
.
.
.
.
.
.
.

.
.
.
.
.
.
.
.
.
.
.
.
.
.
.

.
.
.
.
.
.
.
.
.
.
.
.
.
.
.

.
.
.
.
.
.
.
.
.
.
.
.
.
.
.

.
.
.
.
.
.
.
.
.
.
.
.
.
.
.

.
.
.
.
.
.
.
.
.
.
.
.
.
.
.

.
.
.
.
.
.
.
.
.
.
.
.
.
.
.

.
.
.
.
.
.
.
.
.
.
.
.
.
.
.

.
.
.
.
.
.
.
.
.
.
.
.
.
.
.

.
.
.
.
.
.
.
.
.
.
.
.
.
.
.

.
.
.
.
.
.
.
.
.
.
.
.
.
.
.

.
.
.
.
.
.
.
.
.
.
.
.
.
.
.

43
43
45
47
48
51
53
55
56
59
60
62
63
65
68
69

B Result Tables
B.1 Total . . . . . . . . . . . . . . . . . . . . .
B.1.1 Numeric variable . . . . . . . . .
B.1.2 Categorical variable with 7 levels
B.2 Mean . . . . . . . . . . . . . . . . . . . .
B.2.1 Numeric variable . . . . . . . . .
B.2.2 Categorical variable with 7 levels
B.3 Regression . . . . . . . . . . . . . . . . .
B.4 Quantiles . . . . . . . . . . . . . . . . . .
B.5 Survey Tables . . . . . . . . . . . . . . .
B.6 Total - Replicate Weights . . . . . . . . .
B.7 Mean - Replicate Weights . . . . . . . .
B.8 Histogram . . . . . . . . . . . . . . . . .
B.9 Boxplot . . . . . . . . . . . . . . . . . . .
B.10 Hexagon Binning . . . . . . . . . . . . .

.
.
.
.
.
.
.
.
.
.
.
.
.
.

.
.
.
.
.
.
.
.
.
.
.
.
.
.

.
.
.
.
.
.
.
.
.
.
.
.
.
.

.
.
.
.
.
.
.
.
.
.
.
.
.
.

.
.
.
.
.
.
.
.
.
.
.
.
.
.

.
.
.
.
.
.
.
.
.
.
.
.
.
.

.
.
.
.
.
.
.
.
.
.
.
.
.
.

.
.
.
.
.
.
.
.
.
.
.
.
.
.

.
.
.
.
.
.
.
.
.
.
.
.
.
.

.
.
.
.
.
.
.
.
.
.
.
.
.
.

.
.
.
.
.
.
.
.
.
.
.
.
.
.

.
.
.
.
.
.
.
.
.
.
.
.
.
.

.
.
.
.
.
.
.
.
.
.
.
.
.
.

.
.
.
.
.
.
.
.
.
.
.
.
.
.

.
.
.
.
.
.
.
.
.
.
.
.
.
.

.
.
.
.
.
.
.
.
.
.
.
.
.
.

.
.
.
.
.
.
.
.
.
.
.
.
.
.

.
.
.
.
.
.
.
.
.
.
.
.
.
.

77
77
77
77
78
78
78
79
79
80
80
80
81
81
82

Bibliography

.
.
.
.
.
.
.
.
.
.
.
.
.
.
.

.
.
.
.
.
.
.
.
.
.
.
.
.
.
.

.
.
.
.
.
.
.
.
.
.
.
.
.
.
.

.
.
.
.
.
.
.
.
.
.
.
.
.
.
.

.
.
.
.
.
.
.
.
.
.
.
.
.
.
.

.
.
.
.
.
.
.
.
.
.
.
.
.
.
.

83

xi

List of Figures
3.1
3.2
3.3
3.4
3.5
3.6

svydbhist example . . . . . . . . . . . . . . . . . .
svydbboxplot example . . . . . . . . . . . . . . . .
svydbhexplot example . . . . . . . . . . . . . . . .
Hexagon binning explanation, (Lewin-Koh, 2016)
svydbcoplot example . . . . . . . . . . . . . . . . .
Why use ggplot . . . . . . . . . . . . . . . . . . . .

.
.
.
.
.
.

.
.
.
.
.
.

.
.
.
.
.
.

.
.
.
.
.
.

.
.
.
.
.
.

.
.
.
.
.
.

.
.
.
.
.
.

.
.
.
.
.
.

.
.
.
.
.
.

.
.
.
.
.
.

.
.
.
.
.
.

.
.
.
.
.
.

22
24
26
27
28
29

4.1
4.2
4.3
4.4
4.5
4.6
4.7
4.8
4.9
4.10
4.11
4.12

Population total time comparison - Numeric . .
Population total time comparison - Categorical .
Population mean time comparison - Numeric . .
Population mean time comparison - Categorical
Regression time comparison - 5 variables . . . .
Median time comparison - Categorical . . . . . .
Tables time comparison - Categorical . . . . . . .
Replicate total time comparison - Numeric . . .
Replicate mean time comparison - Numeric . . .
Histogram time comparison - Categorical . . . .
Boxplot time comparison . . . . . . . . . . . . . .
Hexagon Binning time comparison . . . . . . . .

.
.
.
.
.
.
.
.
.
.
.
.

.
.
.
.
.
.
.
.
.
.
.
.

.
.
.
.
.
.
.
.
.
.
.
.

.
.
.
.
.
.
.
.
.
.
.
.

.
.
.
.
.
.
.
.
.
.
.
.

.
.
.
.
.
.
.
.
.
.
.
.

.
.
.
.
.
.
.
.
.
.
.
.

.
.
.
.
.
.
.
.
.
.
.
.

.
.
.
.
.
.
.
.
.
.
.
.

.
.
.
.
.
.
.
.
.
.
.
.

.
.
.
.
.
.
.
.
.
.
.
.

.
.
.
.
.
.
.
.
.
.
.
.

32
32
33
33
34
34
35
35
36
36
37
37

.
.
.
.
.
.
.
.
.
.
.
.

1

Chapter 1

Introduction
1.1

Background

Currently there is already a survey package (Lumley, 2004) in R which is at a stable production status, it provides survey analysis, including graphics, estimation
and inference. It also supports both replicate-weight and Taylor linearisation standard errors, and can efficiently handle multistage stratified designs without replacements. However, it requires the data sets to be stored in a data frame in memory. For
most survey data sets this is not a problem, however, nowadays there are a number
of large survey data sets, for example the American Community Survey (ACS) includes 3,000,000 people per year, and the Nationwide Emergency Department Sample (NEDS) includes more then 25,000,000 hospital visit records per year.
In R, there are currently two approaches to analysis survey data sets in a database.
The first is to use the survey package, with it’s database back-end function, the data
sets can be loaded into memory without any problem, but the time to analysis the
data may not be promising when the data sets are too large.
Another approach is to perform as much computation as possible directly in the
database, so that only small bits of data or numbers are transferred into memory
when necessary. This approach is more efficient but is less flexible, since mathematics and statistical operations are limited in a database. Another advantage of this
approach is that if the database is powerful, then the computation would be faster
than just using a standard laptop or desktop.
The second approach is implemented in the sqlsurvey R package (Lumley, 2014),
however codes which communicates with the databases are written in "hand-written
SQL code". Therefore, it would be hard to maintain and would cause compatible issues with different types of databases.
The better approach to analyse survey data sets in a database would be to use the
R packages dplyr (Wickham et al., 2017) and dbplyr (Wickham and Ruiz, 2018) as a
database interface. Since these packages are maintained by experts at Rstudio, it is
likely that these packages are more stable than others, bug fixes and updates would
also be quick. So, in this project i will implement a sets of functions to analyse survey
data sets with the second approach, named svydb.

1.2

Survey data in SQL

As mentioned in section 1.1, when we are analysing large survey data sets, it would
be more feasible to do it in a database. Some commonly used survey statistics are
survey mean, survey total, summaries and regression.
Survey totals, means and summaries can be easily computed in a database, since
it only requires simple arithmetic like summing, multiplications, divisions along
with some grouping.
Regression may require a bit more work, since it requires matrix operations
which is not supported in SQL, however by loading a few chucks of small matrices
into memory, it can still be easily implemented in a database, since after all regression coefficients and their variances only requires sums and multiplications.
More details of the calculations and difficulties will be discussed later on in
Chapter 2.

1.3
1.3.1

Coding with dplyr and dbplyr
Introduction to dplyr

The dplyr package was implemented to manipulate, clean and construct data. With
this package, data manipulation and data exploration can be done easily and quickly,
since the it was written in a computationally efficient manner.
The package contains a few common data manipulating functions such as selecting specific columns, arranging or creating new columns, filtering rows, merging
data (joins) and summarising data by groups. Other functions such as simple statistics operations are also included in the package.

1.3.2

Pipes

The pipe operator (%>%) first appeared in magrittr package (Bache and Wickham,
2014), and is created to make codes more readable. The pipe operator inputs the
object on the left-hand side of the pipe into the function on right hand side. Some
basic piping are as follows:
• x %>% f is equivalent to f(x).
• x %>% f(y) is equivalent to f(x, y).
It is rather useful when we have multiple steps while we are transforming data sets,
because naturally we read from left to right. For example, with traditional coding,
reading is always from inside out,
> x = sample ( 1 0 )
> summary ( diff ( exp ( floor ( cos ( x )))))
with piping, it is much easier to read,
> x % >% cos () % >% floor () % >% exp () % >% diff ()
% >% summary ()
Though the pipe has its advantages, there are also times that it is not useful. It would
not be useful when the intermediate variables are needed or when the intermediate
variables requires heavy computation.

1.3. Coding with dplyr and dbplyr

1.3.3

3

dplyr’s SQL compatibility (dbplyr)

As mentioned in section 1.3.1, there are six basic functions in dplyr. These functions
are all related to the basic SQL queries.
dplyr Function
select()
filter ()
group_by()
arrange()
join()
mutate()

Description
Selecting columns (variables)
Filter (subset) rows.
Group the data
Sort the data
Joining tables
Creating New Variables (Columns)

Equivalent SQL
SELECT
WHERE
GROUP BY
ORDER BY
JOIN
COLUMN ALIAS

With dbplyr, when these dplyr functions are applied onto a sql table, they automatically translate itself into SQL queries. For example,
> mtdb % >% select ( mpg , gear ) % >% group _ by ( gear ) % >%
summarise ( sum _ mpg = sum ( mpg )) % >% head ( 3 ) % >%
show _ query ()
#
#
#
#
#
#

<SQL >
SELECT " gear " , SUM (" mpg ") AS " sum _ mpg "
FROM ( SELECT " mpg " AS " mpg " , " gear " AS " gear "
FROM " mtcars ") " gaecowztcc "
GROUP BY " gear "
LIMIT 3

With this approach, dplyr does not actually do any work, it’s job is only to translate
the codes into SQL and gives the database instructions. Another advantage of this
method is that the intermediate variables between the pipes only builds up the query
and does not get evaluated nor is stored anywhere.

1.3.4

Quasi-quotation

Programming with dplyr relies on a concept called the Quasi-quotation, also known
as the non standard evaluation, it means that while we are doing some evaluation
with dplyr in R, we are not using R’s standard evaluation method. For example,
with R’s standard method,
test _ func = function (x , y ){
x + y
}
> x1 = 1; x2 = 2
> test _ func ( x 1 , x 2 )
R looks for the variables x1 and x2 in the environment and inputs it into the function
test_func.
However, while programming in dplyr,
> mtcars % >% select ( mpg )
The variable mpg is a variable in the data set and cannot be found in the environment,
it is quoted and evaluated in a non standard way.

4

Chapter 1. Introduction

Though, it might look useful to use this non standard evaluation method, but it
is more difficult to program with, for example while writing a function,
test _ fun 2 = function ( data , x ){
data % >% select ( x ) % >% head ( 2 )
}
> test _ fun 2 ( mtcars , mpg )
# Error : ‘x ‘ must resolve to integer column positions ,
# not a list
Since dplyr does not evaluate in the standard way, we cannot just pass a variable
in like the standard method. We will need to quote it with the quo() or enquo()
function.
test _ fun 2 = function ( data , x ){
x = enquo ( x )
data % >% select (!! x ) % >% head ( 2 ) % >% tbl _ df ()
}
> test _ fun 2 ( mtcars , mpg )
#
mpg
# * <dbl >
#1
21
#2
21
enquo() allows us to quote (also known as quosure) the variable so we can pass it
into the select() function, and !! (bang bang) allows us to unquote the variable at
evaluation.
There are also other similar functions to help us overcome the difficulties while
programming with dplyr, like quos(), sym() and quo_name() which are used in
different situations.

1.3.5

Common issues while coding with dplyr in a database

• No factor types.
• Difficult to code with quasi-quotation.
• Cannot do row-wise operations due to laziness. That is, the data sets within a
database in R will not be loaded into memory unless required.
• No matrix operations.
• No base R functions.
• No distributions.
• Inconsistent restrictions between databases.

5

Chapter 2

Methodology
2.1

Survey Design

When analysing survey data sets in the survey package, a survey design (svydesign()) is always required. The survey design object combines the data set and all
the survey design information needed to analyse it. These objects are used by the
survey modelling and summary functions.
The sets of functions that surveydb provides also adapted that concept but with
a few modifications, it uses the R6 (Chang, 2017) class system which is encapsulated
object orientation programming and is different to the standard S3 and S4 in base
R which uses functional object orientation programming. The main difference between the two is that S3 and S4 methods and objects are separate and in R6, object
contains methods and data.
The advantage of encapsulated object orientation programming is that information within the objects are not computed unless it is needed, for example when the
sum of all sampling weights are needed, we can compute it by using a method
within the object and the value also updates when it’s been called on a subset of
the object.
> nh . dbsurv = svydbdesign ( st = SDMVSTRA , wt = WTMEC 2 YR ,
id = SDMVPSU , data = nhdb )
> nh . dbsurv $ getwt ()
[1] 306590681
> nh . dbsurv $ subset ( Race 3 == 3 )$ getwt ()
[1] 192721267
Therefore, it is much more time efficient than creating a survey design that contains all the information, since the user may not need every information within the
object.
The svydbdesign() function, has four basic arguments,
svydbdesign(st = NULL, id = NULL, wt, data)
• st = Column name specifying the strata column. NULL for no strata.
• id = Column name specifying the cluster column. NULL for no cluster.
• wt = Column name specifying the sampling weights column.
• data = A data frame or sql table of the survey data set.

6

Chapter 2. Methodology

When the svydb.design object is called directly, a brief description of the design
will be shown.
> nh . dbsurv
svydb . design , 9 7 5 6 observation ( s ) , 3 1 Clusters
Functions within the svydb.design object for users to use are,
Classes ’ svydb . design ’ , ’R 6 ’ < svydb . design >
.
.
clone : function ( deep = FALSE )
getmh : function ()
getwt : function ()
subset : function (... , logical = T )
subset _ rows : function ( from , to )
• clone() = Create a clone of the object.
• getmh() = A table indicating strata and cluster information.
• getwt() = Compute the sum of the sampling weights.
• subset() = Subset the design by conditions, similar to base :: subset.
e.g.(design$subset(Race3 == 3)
• subset_rows() = Subset the design by rows. e.g.(design$subset_rows(1, 100)
)

2.2. Population Total

2.2

7

Population Total

The function svydbtotal() was designed to estimate the population total in R by
using dplyr, it is compatible with data frames and sql tables and was designed to do
as much computation as possible in a database.
In the function svydbtotal(), the total is computed by using the Horvitz-Thompson
estimator (Horvitz and Thompson, 1952), it is an unbiased estimator of the population total.
ˆ =
Total

mh

L

∑ ∑ zhi

h =1 i =1

where,
zhi =

∑

whij xhij

j∈ PSU

• L = number of stratum
• mh = number of clusters in stratum h
• whij = the sample weight in stratum h and cluster i for observation j.
Variance estimation of the total uses the Horvitz-Thompson estimator on an influence function
ˆ )=
Var ( Total

m h mh
∑ mh − 1 ∑ (zhi − z̄h )T (zhi − z̄h )
i =1
h =1
L

where,
1
z̄h =
mh

2.2.1

mh

∑ zhi

i =1

Usage
svydbtotal(x, num = T, return.total = F, design)

2.2.2

Arguments

• x = Name indicating the variable.
• num = TRUE or FALSE indicating whether x is numeric or categorical.
• return.total = TRUE to return only totals, no standard errors.
• design = svydb.design object.

8

Chapter 2. Methodology

2.2.3

Examples

> nh . dbsurv = svydbdesign ( st = SDMVSTRA , wt = WTMEC 2 YR ,
id = SDMVPSU , data = nhdb )
> svydbtotal ( x = Race 3 , design = nh . dbsurv , num = T )
#
Total
SE
# Race 3 9 5 9 3 8 0 8 4 2 6 1 4 3 2 5 9 5
>
#
#
#
#
#
#
#

svydbtotal ( x = Race 3 , design = nh . dbsurv , num = F )
Total
SE
Race 3 _ 1 2 9 8 1 2 3 1 6 6 1 1 2 5 2 7
Race 3 _ 2 2 1 4 1 6 1 6 4 4 4 8 5 8 6 5
Race 3 _ 3 1 9 2 7 2 1 2 6 7 2 3 4 3 1 2 9 6
Race 3 _ 4 3 8 1 3 1 5 3 8 5 5 6 1 1 6 1
Race 3 _ 6 1 5 5 1 9 5 2 9 2 3 6 7 7 2 3
Race 3 _ 7
8989867 1468813

> svydbtotal ( x = Race 3 , design = nh . dbsurv , num = T ,
return . total = T )
#
Total
# Race 3 9 5 9 3 8 0 8 4 2
Generic functions like coef() and SE() were also implemented to extract the
coefficients and standard errors from a svydbstat object.
> class ( svydbtotal ( x = Race 3 , design = nh . dbsurv , num = T ))
# [ 1 ] " svydbstat "
> coef ( svydbtotal ( x = Race 3 , design = nh . dbsurv , num = T ))
# [1] 959380842
> SE ( svydbtotal ( x = Race 3 , design = nh . dbsurv , num = T ))
#
Race 3
# 61432595

2.2.4

Difficulties

1. In SQL if a column contains only numbers, it is not possible to identify whether
a column type is factor or numeric, therefore in the svydbtotal() function, the
user needs to specify it. num = TRUE: Numeric, num = FALSE: Factor.
2. To compute the population total for a categorical variable, dummy variables
are needed. In SQL, there are two ways to do this. The first way would be
to create new columns for every levels of the variable manually and apply
ifelse/CASE WHEN to each of the columns. Another way would be to create
a small contrast table in memory, and use INNER JOIN to join the small table
onto the data set. There second approach is much faster, and the function
dummy_mut() adapts that approach. It creates (mutate) new dummy columns
on the right side of the data set.
3. Calculating the variance/standard error is the most complex part of svydbtotal().
Therefore the svyVar() function is written to calculate the variance of a variable. If the variable is a categorical variable with multiple levels, the calculations will be replicated with sapply().

2.3. Population Mean

2.3

9

Population Mean

The function svydbmean() was designed to estimate the population mean in R by
using dplyr, it is compatible with data frames and sql tables and was designed to do
as much computation as possible in a database.
In the function svydbmean(), the mean is computed by using the Horvitz-Thompson
estimator, it is an unbiased estimator of the population mean.
ˆ = 1
Mean
N̂

L

mh

∑ ∑ zhi

h =1 i =1

where,
N̂ =

∑

wj,

zhi =

j∈ PSU

∑

whij xhij

j∈ PSU

• L = number of stratum
• mh = number of clusters in stratum h
• whij = the sample weight in stratum h and cluster i for observation j.
Variance estimation of the mean uses the Horvitz-Thompson estimator on an influence function
ˆ )=
Var ( Mean

m h mh
∑ mh − 1 ∑ (dhi − d¯h )T (dhi − d¯h )
i =1
h =1
L

where,
dhi =

2.3.1

1
N̂

∑

whij ( xhij − x̄ ),

j∈ PSU

1
d¯h =
mh

mh

∑ dhi

i =1

Usage
svydbmean(x, num = T, return.mean = F, design)

2.3.2

Arguments

• x = Name indicating the variable.
• num = TRUE or FALSE indicating whether x is numeric or categorical.
• return.mean = TRUE to return only means, no standard errors.
• design = svydb.design object.

10

Chapter 2. Methodology

2.3.3

Examples

> nh . dbsurv = svydbdesign ( st = SDMVSTRA , wt = WTMEC 2 YR ,
id = SDMVPSU , data = nhdb )
> svydbmean ( x = Race 3 , design = nh . dbsurv , num = T )
#
Mean
SE
# Race 3 3 . 1 2 9 2 0 . 0 6 7 4
>
#
#
#
#
#
#
#

svydbmean ( x = Race 3 , design = nh . dbsurv , num = F )
Mean
SE
Race 3 _ 1 0 . 0 9 7 2 3 8 0 . 0 2 0 8
Race 3 _ 2 0 . 0 6 9 8 5 3 0 . 0 1 5 4
Race 3 _ 3 0 . 6 2 8 5 9 5 0 . 0 4 0 7
Race 3 _ 4 0 . 1 2 4 3 7 3 0 . 0 2 3 9
Race 3 _ 6 0 . 0 5 0 6 2 0 0 . 0 0 8 0
Race 3 _ 7 0 . 0 2 9 3 2 2 0 . 0 0 4 5

Generic functions like coef() and SE() were also implemented to extract the
coefficients and standard errors from a svydbstat object. This is useful since the
svydbstat objects are rounded when they are printed, by using coef() and SE(), the
unrounded value can be extracted.
> class ( svydbmean ( x = Race 3 , design = nh . dbsurv , num = T ))
# [ 1 ] " svydbstat "
> coef ( svydbmean ( x = Race 3 , design = nh . dbsurv , num = T ))
# [1] 3.129191
> SE ( svydbmean ( x = Race 3 , design = nh . dbsurv , num = T ))
#
Race 3
# 0.06735437

2.3.4

Difficulties

1. In SQL cannot recognise factor variables. Explained in difficulty 1 from chapter
2.2 (Population Total).
2. SQL cannot create dummy variables. Explained in difficulty 2 from chapter 2.2
(Population Total).
3. Difficult to compute the variances. Explained in difficulty 3 from chapter 2.2
(Population Total).

2.4. Regression

2.4

11

Regression

The function svydblm() was designed to fit a linear model to survey data in R by
using dplyr, it is compatible with data frames and sql tables and was designed to do
as much computation as possible in a database.
In the function svydblm(), the coefficients are computed by,
β̂ = ( X T WX )−1 ( X T WY )
• W = Sampling weights
Variance estimation of the coefficients uses a similar approach as survey mean/total,
Var pq ( β̂ˆ ) =

m h mh
∑ (zhip − z̄hp )T (zhiq − z̄hq )
m
−
1
h
i =1
h =1
L

∑

where,
zhi = xhij whij (yhij − µhij ),

z̄h =

1
mh

mh

∑ zhi

i =1

And by using the sanwich estimator,
cov( β̂) = ( X T WX )−1 Var pq ( β̂)( X T WX )−1
• L = number of stratum
• mh = number of clusters in stratum h
• whij = the sample weight in stratum h and cluster i for observation j.
• p, q = Indicator function for variables p and q.

2.4.1

Usage
svydblm(formula, design)

2.4.2

Arguments

• f ormula = Model formula.
• design = svydb.design object.

12

Chapter 2. Methodology

2.4.3

Examples

> nh . dbsurv = svydbdesign ( st = SDMVSTRA , wt = WTMEC 2 YR ,
id = SDMVPSU , data = nhdb )
> svydblm ( DirectChol ~ Age + BMI + factor ( Gender ) ,
design = nh . dbsurv )
# svydb . design , 9 7 5 6 observation ( s ) , 3 1 Clusters
#
# Survey design :
# svydbdesign ( st = SDMVSTRA , id = SDMVPSU , wt = WTMEC 2 YR ,
#
data = nhdb )
#
# Call :
# svydblm ( formula = DirectChol ~ Age + BMI + factor ( Gender ) ,
#
design = nh . dbsurv )
#
# Coefficients :
#
intercept Age
BMI
Gender _ 2
# [ 1 ,]
1.632111
0 . 0 0 3 2 5 4 -0 . 0 1 8 6 3 6
0.218086
The summary() function to obtain the summary of the model and predict() function to predict using new data-sets with the model was also implemented,
# dbfit = svydblm ( formula = DirectChol ~ Age + BMI ,
#
design = nh . dbsurv )
>
#
#
#
#
#
#
#
#
#
#
#
#
#
#
#
#

summary ( dbfit )
Call :
svydblm ( formula = DirectChol ~ Age + BMI ,
design = nh . dbsurv )

Coefficients :
Estimate Std . Error t value Pr ( >| t |)
intercept 1 . 7 2 6 3 1 3 5 0 . 0 2 7 9 7 0 3 6 1 . 7 1 9 < 2e - 1 6 ***
Age
0.0034161 0.0003428
9 . 9 6 6 5 . 2 3e - 0 8 ***
BMI
-0 . 0 1 8 2 4 6 8 0 . 0 0 1 1 2 7 7 -1 6 . 1 8 1 6 . 6 3e - 1 1 ***
--Signif . codes : 0
***
0.001
**
0.01
*
0.05
.
0.1
1

>
#
#
#
#

predict ( dbfit , newdata = data . frame ( Age = 1 : 3 , BMI = 4 : 6 ))
link
SE
1 1.6567 0.0242
2 1.6419 0.0233
3 1.6271 0.0225

Survey design :
svydbdesign ( st = SDMVSTRA , id = SDMVPSU , wt = WTMEC 2 YR ,
data = nhdb )

2.4. Regression

13

Other generic function includes coef(), SE(), vcov() and residuals().
# dbfit = svydblm ( formula = DirectChol ~ Age + BMI ,
#
design = nh . dbsurv )
> coef ( dbfit )
#
intercept
Age
BMI
# [ 1 ,] 1 . 7 2 6 3 1 4 0 . 0 0 3 4 1 6 1 2 4 -0 . 0 1 8 2 4 6 7 6
> SE ( dbfit )
#
intercept
Age
BMI
# 0.0279703171 0.0003427746 0.0011276877
>
#
#
#
#

vcov ( dbfit )
intercept
Age
BMI
intercept 7 . 8 2 3 3 8 6e - 0 4 3 . 7 6 9 6 6 3e - 0 6 -2 . 8 0 6 3 4 0e - 0 5
Age
3 . 7 6 9 6 6 3e - 0 6 1 . 1 7 4 9 4 4e - 0 7 -2 . 4 9 6 4 0 3e - 0 7
BMI
-2 . 8 0 6 3 4 0e - 0 5 -2 . 4 9 6 4 0 3e - 0 7 1 . 2 7 1 6 8 0e - 0 6

>
#
#
#
#
#
#
#

head ( residuals ( dbfit ) , 3 )
Source :
lazy query [?? x 1 ]
Database : Mo ne tDB Em be dde dC on nec ti on
residuals
<dbl >
1
-0 . 3 1 6
2
-0 . 3 1 8
3
-0 . 7 3 3

2.4.4

Difficulties

1. SQL cannot create dummy variables. Explained in difficulty 2 from chapter 2.2
(Population Total).
2. In SQL matrix multiplications are not supported, however, it can still be implemented since matrix multiplications only requires addition and multiplication.
For example with matrix X, by calculating the sums of products of the first column of the matrix with the rest of the columns (including the first column), we
get the first row of the X T X matrix, and to get the whole X T X matrix we only
need to repeat the process with different columns. However, the inverse of
a matrix is not possible in SQL, so to compute ( X T X )−1 , we need to pull the
matrix into memory.
3. The variance for the regression coefficients are a bit more complicated than
computing the variance for mean/total, since for mean/total we only need the
diagonal of the co-variance matrix. However, the variance of the regression coefficients requires the whole co-variance matrix. This means that more replication with different combinations of zhip /zhiq will be needed, but we only need
the upper triangle or the lower triangle of the matrix, since it is a symmetric
matrix.

14

Chapter 2. Methodology

2.5

Quantiles

The function svydbquantile() was designed to compute the medians/quantiles
from survey data in R by using dplyr, it is compatible with data frames and a few
types of sql tables and was designed to do as much computation as possible in a
database.
To estimate the median/quantiles, a standard probabilistic algorithm is used. For
example to estimate the median,
1. Take a sample of size n2/3 from the data set. (Proof below)
2. Compute the 99% confidence interval [ a, b] of the 0.5 quantile by using svyquntile()
from the survey package.
3. Read in the data set where the observations are between a and b.
4. Sort the data and compute the cumulative sum of the weights of the read in
observations, wn = ∑in=1 wi
5. Find out if the median is within the read in data set. Since median = 0.5 × W,
we can find out by searching if median equals or is between wn .
6. If the median is not found, repeat.
Other quantiles are computed with the same method.
Proof:
Proof. Let M be a random sample of N.
Then it’s confidence interval length is ∝ M−1/2 .
Though by using this method to compute the survey quantile requires at least
two sets of data to be passed into memory, however it is still much more efficient
than sorting and calculating the cumulative sum for the whole data set, since n2/3 is
relatively small compared to whole data set (One million observations, 100000002/3 =
10000).

2.5. Quantiles

2.5.1

15

Usage
svydbquantile(x, quantiles = 0.5, design)

2.5.2

Arguments

• x = Name indicating the variable.
• quantiles = Quantiles to estimate, a number, or a vector of numbers for multiple
quantiles. Default to 0.5.
• design = svydb.design object.

2.5.3

Examples

> db . dbsurv = svydbdesign ( st = SDMVSTRA , wt = WTMEC 2 YR ,
id = SDMVPSU , data = nhdb )
> svydbquantile ( x = Age , quantile = 0 . 5 , design = nh . dbsurv )
# 0.5
# 37
> svydbquantile ( x = BMI , quantile = c ( 0 . 2 5 ,0 . 7 5 ) ,
design = nh . dbsurv )
# 0.25 0.75
# 21.7 30.6

2.5.4

Difficulties

1. To compute the survey quantile, a sample of the data-set is needed. However, different types of SQL uses different queries for sampling tables and the
sample_frac() function from dplyr currently only works with Spark (Luraschi
et al., 2018) database connections and local dataframes/tibbles. Sampling in
MonetDB (MonetDB-B.V., 2008) was also implemented. Therefore, currently
the svydbquantile() is only tested with local dataframes and database connections that are mentioned above. It may be possible that it is compatible
with other types of connections but they are to be tested. Another possibility
is that the sample_frac() function will be extended in the future to support
more database connections.
2. Since to compute the survey quantile in svydbquantile(), the inputted dataset will be sized down into n2/3 , so if the inputted data-set is a pre-subsetted
data-set then it means that the data-set will be subsetted at least twice to compute the quantiles. This could be a problem because svydbquantile() runs
through svyquantile() form the survey package which uses svymean() within
it. If the data-set is too small it may lose enough information and may cause
mathematical errors. For example, if there are only one cluster within a strata
then it will cause the equation to divide by zero.
To overcome this, there is a option within the survey package called "survey.loney.psu", if we set this option to "adjust", e.g. options("survey.lonely.psu"
= "adjust"), it will allow survey statistic computations even if there are only
one cluster within a strata.

16

Chapter 2. Methodology

2.6

Survey Tables

The function svydbtable() was designed to create contingency tables for survey
data, it is compatible with data frames and sql tables and was designed to do as
much computation as possible in a database.
Each cell within the table is computed with the same method as svydbtotal().

2.6.1

Usage
svydbtable = function(formula, design, as.local = F)

2.6.2

Arguments

• f ormula = A formula specifying margins for the table, only + can be used.
• design = svydb.design object.
• as.local = A logical value indicating the returning object type. Default is database
tables, tbl_sql.

2.6.3

Examples

> nh . dbsurv = svydbdesign ( st = SDMVSTRA , wt = WTMEC 2 YR ,
id = SDMVPSU , data = nhdb )
>
#
#
#
#
#
#
#
#
#
#
#

svydbtable (~ MaritalStatus , design = nh . dbsurv )
Source :
lazy query [?? x 2 ]
Database :
M one tD BE mbe dd ed Con ne cti on
Ordered by : MaritalStatus
MaritalStatus
wt
<int >
<dbl >
1
1 118752657.
2
2 12600347.
3
3 23868539.
4
4
5486968.
5
5 44543092.
6
6 18664186.

> svydbtable (~ MaritalStatus , design = nh . dbsurv ,
as . local = T )
#
MaritalStatus
wt
#
<int >
<dbl >
# 1
1 118752657.
# 2
2 12600347.
# 3
3 23868539.
# 4
4
5486968.
# 5
5 44543092.
# 6
6 18664186.

2.6. Survey Tables
> svydbtable (~ Race 3 + Smoke 1 0 0 , design = nh . dbsurv ,
as . local = T )
# A tibble : 6 x 5
#
Race 3 Smoke 1 0 0 _ 1 Smoke 1 0 0 _ 2 Smoke 1 0 0 _ 7 Smoke 1 0 0 _ 9
#
<int >
<dbl >
<dbl >
<dbl >
<dbl >
# 1
1
6177437. 11124548.
0
0
# 2
2
5375656.
9313414.
0
0
# 3
3 71234238. 77533263.
0
29247.
# 4
4
9687657. 16031251.
22142.
14686.
# 5
6
3019921.
8645276.
0
14966.
# 6
7
3070772.
2650765.
0
0

> svydbtable (~ Race 3 + Work + Gender , design = nh . dbsurv ,
as . local = T )
#
#
#
#
#
#
#
#
#
#

$ ‘ Gender = 1 ‘
A tibble : 6 x 6
Race 3
Work _ 1
Work _ 2
Work _ 3
Work _ 4 Work _ 7
<int >
<dbl >
<dbl >
<dbl >
<dbl > <dbl >
1
1 7063412. 118589. 805525. 2148047. 54555.
2
2 5066507.
50966. 283436. 2056117.
0
3
3 48916000. 1670881. 3052152. 22756262.
0
4
4 6557730. 166729. 812680. 5201428.
0
5
6 3689443.
79793. 191406. 1764683.
0
6
7 1747070.
34664. 258597. 1182825.
0

#
#
#
#
#
#
#
#
#
#

$ ‘ Gender = 2 ‘
A tibble : 6 x 6
Race 3
Work _ 1
Work _ 2
Work _ 3
Work _ 4 Work _ 7
<int >
<dbl >
<dbl >
<dbl >
<dbl > <dbl >
1
1 4839422.
80842. 273991. 4040679.
0
2
2 4052770.
20057. 400463. 3907392.
0
3
3 42775016. 1370755. 2995602. 34142499.
0
4
4 7486159. 280862. 915095. 7070385.
0
5
6 3557006. 111248. 209497. 2803386.
0
6
7 1810965.
17984. 114653. 1255186.
0

17

18

Chapter 2. Methodology

2.7

Survey Statistic on Subsets

The function svydbby() was designed to compute survey statistics on subsets of the
data in R by using dplyr, it is compatible with data frames and sql tables and was
designed to do as much computation as possible in a database.
This function creates a number of subsets based on the conditions given by the
user and computes the desired survey statistic on all the subsets. Currently, it is only
compatible with svydbtotal() and svydbmean().

2.7.1

Usage
svydbby(x, by, FUN, design, ...)

2.7.2

Arguments

• x = A variable specifying the variable to pass to FUN.
• by = A variable specifying factors that define the subsets.
• FUN = A function indicating the desired survey statistics.
• design = svydb.design object.
• ... = Other arguments to pass to FUN.

2.7.3

Examples

> nh . dbsurv = svydbdesign ( st = SDMVSTRA , wt = WTMEC 2 YR ,
id = SDMVPSU , data = nhdb )
> svydbby ( x = Age , by = Gender , FUN = svydbmean ,
design = nh . dbsurv , num = T )
# $ Age
#
Mean
SE
# Gender == 1 3 6 . 2 1 0 3 5 0 . 8 3 8 7 4 5 9
# Gender == 2 3 8 . 0 9 8 9 9 0 . 6 7 2 1 7 8 9
> svydbby ( x = BMI , by = Race 3 , FUN = svydbtotal ,
design = nh . dbsurv , num = T )
# $ BMI
#
Total
SE
# Race 3 == 3 5 0 0 4 8 2 2 1 0 0 6 1 2 1 8 8 9 6 6
# Race 3 == 1 7 3 7 9 3 9 0 5 7 1 5 3 5 5 1 6 6 2
# Race 3 == 6 3 4 7 4 3 5 9 0 2 5 2 8 7 6 4 6 4
# Race 3 == 4 1 0 2 0 4 7 6 2 2 1 1 5 4 6 3 0 7 9 3
# Race 3 == 7 2 2 7 7 2 9 7 2 4 3 9 6 3 7 0 0 6
# Race 3 == 2 5 5 0 2 0 7 4 8 4 1 1 8 3 3 1 9 4 1

2.8. Replicate Weights

2.8

19

Replicate Weights

In survey data sets, standard errors can never be known with any certainty and are
only estimated. Replicate weights lets us to use a single sample with different sampling weights to capture the characteristics of multiple samples, it lets us compute
more informed standard error estimates. Though computing standard errors from
replicate weights usually result in them getting bigger, but the increase usually is
not large enough that it can alter the significance level.
Another reason for us to use replicate weights is that it provides a less complex
way to compute the standard errors.
Currently, replicate weights are available in a number of data sets, for example
the American Community Survey and Puerto Rican Community Survey data. In
these data sets, there are 80 separate replicate weights at the household/person level
which allows us to compute the standard errors.

2.8.1

Replicate Standard Errors

To compute the replicate standard errors, there are three steps.
1. Compute the survey statistics of interest with the full sample weights.
2. Rerun the analysis with each sets of the replicate weights.
3. Calculate the standard error,
v
u n (r )
u
SE( X ) = ts ∑ ( X − Xr )2
r =1

• X = Result of the survey statistics using the full sample weights.
• Xr = Result of the survey statistics using the r’th set of the replicate weights.
• s = Scale multiplier. i.e.

2.8.2

4
80

for the American Community Survey.

Survey Replicate Design

Similarly, there is a survey replicate design like the survey design from section 2.1.
Currently, replicate statistics that are supported are survey totals and means.
Therefore, there is no need to provide the stratification and clustering information
to svydbrepdesign().
svydbrepdesign(wt, repwt, scale, data)
• wt = Column name specifying the sampling weights column.
• repwt = A regular expression that matches the names of the replication weight
variables.
• data = A data frame or sql table of the survey data set.

20

2.8.3

Chapter 2. Methodology

Examples

> hde . repsurv = svydbrepdesign ( wt = WGTP , repwt = " wgtp [ 0 -9 ]+ " ,
scale = 4 / 8 0 , data = ss 1 6 hde )
> hde . repsurv
# svydb . repdesign , 4 5 8 2 observation ( s ) ,
#
8 0 sets of replicate weights , scale = 0 . 0 5

2.8.4

Population Total with replicate weights

need to fix
Arguments are the same as svydbtotal().
> hde . dbrepsurv = svydbrepdesign ( wt = WGTP ,
repwt = " wgtp [ 0 -9 ]+ " , scale = 4 / 8 0 , data = ss 1 6 hde )
> svydbreptotal ( x = BATH , design = hde . dbrepsurv , num = T )
> svydbreptotal ( x = FS , design = hde . dbrepsurv , num = F )
> svydbreptotal ( x = HHT , design = hde . dbrepsurv , num = T ,
return . replicates = T )$ replicates
> coef ( svydbreptotal ( x = BATH , design = hde . dbrepsurv ,
num = T ))
> SE ( svydbreptotal ( x = BATH , design = hde . dbrepsurv ,
num = T ))

2.8.5

Population Mean with replicate weights

Arguments are the same as svydbmean().
> hde . dbrepsurv = svydbrepdesign ( wt = WGTP ,
repwt = " wgtp [ 0 -9 ]+ " , scale = 4 / 8 0 , data = ss 1 6 hde )
> svydbrepmean ( x = BATH , design = hde . dbrepsurv , num = T )
> svydbrepmean ( x = FS , design = hde . dbrepsurv , num = F )
> svydbrepmean ( x = HHT , design = hde . dbrepsurv , num = T ,
return . replicates = T )$ replicates
> coef ( svydbrepmean ( x = BATH , design = hde . dbrepsurv ,
num = T ))
> SE ( svydbrepmean ( x = BATH , design = hde . dbrepsurv ,
num = T ))

21

Chapter 3

Graphics
Graphs are useful in statistics because it allows us to have a visual perspective of the
data sets or to present them in a more meaningful way. However, when implementing graphics with survey data sets, it may become more difficult because the data
sets are often large. For example, if every point of the large data set is plotted in a
scatter plot, there will be too many points in the graph which will overlap and hide
each other. There are two ways to over come this problem,
1. Use different symbols, colour or sizes to represent different points or group of
points. This is only suitable for data sets that are moderately large. If the data
sets are too large, the points will still overlap each other.
2. Condense multiple points into one point with an algorithm and use colours to
represent the density of the sampling weights within the point. This will be
discussed with more detail in section 3.3.
Another approach would be to take a sample from the full data set based on the
estimated population distribution and plot the sample.
Currently, there are very few R packages which allow us to plot graphs with
data sets which are stored in a database. One of them is dbplot (Ruiz, 2018), it is
simple and light-weighted but it is in it’s early development stage. As for survey
data sets, except for the surveysql package mentioned in section 1.1, there are no
packages that will allow us to plot data sets with sampling weights without having
the data sets stored in memory.
Like the survey package, svydb also provides a set of tools for plotting survey
data sets and is implemented with ggplot2 (Wickham, 2009), but with less flexibility and options. Similar to the previous chapters, every graphical functions within
svydb is designed to do as much computation in the database as possible.
For some type of graphs, it is reasonable to produce them with data sets outside of memory, since to plot them, we don’t need the whole data set in memory
(computing the statistics for the graphs within the database may even be faster than
transferring the whole data set into memory). For example, plotting a histogram
with one variable, the only information that are needed to plot it is the location of
the breaks for the x-axis and the density of the corresponding break for the y-axis.
Say if there are 30 breaks, then at most, we will only need 60 numbers in memory, 30
for the breaks and 30 for the density.
An example of a graph that is not possible to plot without having the whole data
set in memory is scatter plot, since we will need to know the location of all the points.

22

Chapter 3. Graphics

3.1

Histogram

Traditionally, the heights of each bin for a histogram while plotting the frequency,
the height of each bin is determined by the number of values that are within the
ranges of a certain bin, and plotting the density is determined by the proportion of
data in each bin divided by the width of each bin.
To calculate the proportions, we first need to determine which bin each value is
in. To do this we can use the base::cut() and perform svydbmean() on the cuts.
However this function is not compatible with database tables, therefore a new function db_cut2 was implemented to overcome this difficulty, this will be discussed
with more detail in section 3.1.4.

3.1.1

Usage

svydbhist(x, design, binwidth = NULL, xlab = "x", ylab = "Density")

3.1.2

Arguments

• x = Name indicating the variable.
• design = svydb.design object.
• binwidth = The width of each bin. Binswidths are calculated with Sturges’
formula ref by default, k = [log2 n] + 1..
• xlab, ylab = labels for xlab and ylab.

3.1.3

Examples

> nh . dbsurv = svydbdesign ( st = SDMVSTRA , wt = WTMEC 2 YR ,
id = SDMVPSU , data = nhdb )
> svydbhist ( x = DirectChol , design = nh . dbsurv ,
binwidth = 0 . 2 5 )

F IGURE 3.1: svydbhist example

3.1. Histogram

3.1.4

23

Difficulty

1. In R, it is simple to divide numeric columns into intervals by using base::cut(),
but this function is not supported in SQL. To overcome this problem a new
function was written, db_cut2, it is designed be compatible with SQL tables
and is a simple version of base::cut().

24

Chapter 3. Graphics

3.2

Box Plot

Boxplots are based on the quantiles of each variables or groups of variables. Traditionally, by default in R, the boxes of the boxplot only uses the median, 25th percent
and 75th percent quantile. The ends of the boxplots only extends out to the observations that are within the 1.5 × interquantile range, the rest of the observation outside
this range are plotted as points. In svydbboxplot it is the opposite, due to effiency.

3.2.1

Usage

svydbboxplot(x, groups = NULL, design, varwidth = F, outlier = F)

3.2.2

Arguments

• x, groups = If groups is defined, boxes of x will be split by groups.
• design = svydb.design object.
• varwidth = If varwidth = T, the width of the boxes will be proportional to the
number of observations in that box.
• outlier = If outlier = T, Any observations above or below the 1.5IQR will be
plotted as points.
• all.outlier = TRUE to plot all the outlier points, default FALSE.

3.2.3

Examples

> nh . dbsurv = svydbdesign ( st = SDMVSTRA , wt = WTMEC 2 YR ,
id = SDMVPSU , data = nhdb )
> svydbboxplot ( x = Weight , groups = Race 3 ,
design = nh . dbsurv , outlier = T ,
all . outlier = F , varwidth = T )

F IGURE 3.2: svydbboxplot example

3.3. Hexagon Binning

3.3

25

Hexagon Binning

As mentioned in the beginning of this chapter, since survey data sets can be large
and scatter plots does not work well with large data sets because the points will
hide or overlap each other. A solution would be to use hexagon binning (Carr et al.,
1987), group points that are close to each other into a hexagon and use colours or
sizes of the hexagons to represent the sampling weights of all the points in every
hexagon. The reasons of using hexagons will not be discussed here, but it is proven
to be efficient and less biased visually.
Briefly, the hexagon binning algorithm works as follows,
1. Figure 3.4 (Page 27) Panel 1 - The red point represents the point to be binned.
Blue and black points represents the dual lattice, each point represents the corner of each rectangle.
2. Figure 3.4 (Page 27) Panel 2 - Figure out which rectangles are closest to the red
point. The intersect of the two rectangles should contain the red point.
3. Figure 3.4 (Page 27) Panel 3 - Figure out which hexagon the red point should be
in by calculating the distant between the point and the centres of two hexagons.
The centre for the blue hexagon is the top left corner of the black rectangle and
the centre for the black hexagon is the bottom right corner of the blue rectangle.
4. Repeat for the next point.

3.3.1

Usage/Arguments

To compute the hexagon bins:
svydbhexbin(formula, design, xbins = 30, shape = 1)
• f ormula = A formula indicating x and y. i.e. y x.
• design = svydb.design object.
• xbins = Number of bins on range of the x-axis.
• shape = plotting region, shape = height of y/width of x.
To plot the hexagon bins:
svydbhexplot(d, xlab = d$xlab, ylab = d$ylab)
• d = returning object of svydbhexbin().
• xlab, ylab = labels for x and y axis.

26

Chapter 3. Graphics

3.3.2

Examples

> nh . dbsurv = svydbdesign ( st = SDMVSTRA , wt = WTMEC 2 YR ,
id = SDMVPSU , data = nhdb )
> hb = svydbhexbin ( Height ~ Weight , design = nh . dbsurv )
> svydbhexplot ( hb )

F IGURE 3.3: svydbhexplot example

3.3.3

Difficulties

1. The original code for hexagon binning in the hexbin R package (Carr et al.,
2018) used for-loops to run the algorithm, but it is not possible in SQL. Therefore, new columns and extra calculations were needed to overcome this problem. This caused a efficiency problem.

3.3. Hexagon Binning

F IGURE 3.4: Hexagon binning explanation, (Lewin-Koh, 2016)

27

28

Chapter 3. Graphics

3.4

Conditional Plots

Conditional plots in svydb products sets of hexagon binning graphs conditioned by
the condition given by the user, both x and y axis will remain the same for all graphs.
Currently it only supports conditions applied on factors.

3.4.1

Function description

The svydbcoplot() function, has three basic arguments,
svydbcoplot(formula, by, design)
• f ormula = Formula indicating x and y. i.e. y x.
• by = Formula indicating the conditions of each plot. i.e by1 by2.
• design = svydb.design object.

3.4.2

Examples

> nh . dbsurv = svydbdesign ( st = SDMVSTRA , wt = WTMEC 2 YR ,
id = SDMVPSU , data = nhdb )
> svydbcoplot ( Age ~ Height , by = SmokeNow ~ Gender ,
design = nh . dbsurv )

F IGURE 3.5: svydbcoplot example

3.5. Why ggplot?

3.5

29

Why ggplot?

There are several reasons why graphics in svydb is implemented with ggplot2, one
of them is that ggplot2, dplyr and dbplyr is maintained by the same group of people
and is a part of a project at Rstudio. Though currently, ggplot2 is only compatible
with data sets in memory, but it is possible this will change in the future. Another
reason could be that the user would like to have a interactive plot, this done by
plotly (Sievert et al., 2017).
However, the main reason is that it provides more options for the users in terms
of customising the plots. In base graphics, it would be difficult to customise plots
when the arguments provided does not include what we want, also it would be difficult for the author to consider all the arguments when creating a function.
A simple example is that, when a gg or a ggplot object is created,
> p = ggplot ( mpg , aes ( class , hwy ))
Customising on the object is just as simple as adding other functions on,
> p + geom _ boxplot ( varwidth = T ) +
geom _ jitter ( width = 0 . 2 ) + coord _ flip () +
ggtitle ( " plot " ) + theme _ bw ()

F IGURE 3.6: Why use ggplot

Or switching the style of the plot is as simple as changing the geom,
> p + geom _ violin ()

31

Chapter 4

Results
In this chapter, the time it takes to compute each survey statistics with svydb will be
tested against the survey package, both data sets in memory (local) and in database
will be used for svydb. Up to two million observations was tested with data sets
in memory and up to four million observations for data sets in a data base, data set
used was The National Health and Nutrition Examination Survey.
All computation was done with the same computer, specs as follows,
• MacBook Pro Early 2015
• Processor 2.7 GHz Intel Core i5
• Memory 8 GB 1867 MHz DDR3
• Graphics Intel Iris Graphics 6100 1536 MB
Data sets from 250,000 to 2,000,000 observations were tested on functions which
computes replicate survey statistics, up to 4,000,000 observations for other survey
statistics and graphics. For all the times, please refer to Appendix B.
Legend on graphs corresponds to,
• Survey - Survey package on local data sets.
• svydb.local - svydb on local data sets.
• svydb.database - svydb package on data sets stored in a database.

32

4.1
4.1.1

Chapter 4. Results

Total
Numeric variable

F IGURE 4.1: Population total time comparison - Numeric

4.1.2

Categorical variable with 7 levels

F IGURE 4.2: Population total time comparison - Categorical

4.2. Mean

4.2
4.2.1

33

Mean
Numeric variable

F IGURE 4.3: Population mean time comparison - Numeric

4.2.2

Categorical variable with 7 levels

F IGURE 4.4: Population mean time comparison - Categorical

34

4.3

Chapter 4. Results

Regression

F IGURE 4.5: Regression time comparison - 5 variables

4.4

Quantiles

F IGURE 4.6: Median time comparison - Categorical

4.5. Survey Tables

4.5

35

Survey Tables

F IGURE 4.7: Tables time comparison - Categorical

4.6

Total - Replicate Weights

F IGURE 4.8: Replicate total time comparison - Numeric

36

4.7

Chapter 4. Results

Mean - Replicate Weights

F IGURE 4.9: Replicate mean time comparison - Numeric

4.8

Histogram

F IGURE 4.10: Histogram time comparison - Categorical

4.9. Boxplot

4.9

37

Boxplot

F IGURE 4.11: Boxplot time comparison

4.10

Hexagon Binning

F IGURE 4.12: Hexagon Binning time comparison

38

4.11

Chapter 4. Results

Time Comparison

In general, using svydb with local data sets is always faster than both svydb with
data sets in a database or with the survey package, except for computing the survey
tables (Figure 4.7) and hexagon binning (Figure 4.12), where the times were similar
for survey tables and hexagon binning was always about 0.5 seconds slower.
In terms of data sets in a database, If using svydb with local data sets is faster
than the survey package then the time it takes to compute survey statistics in a
database with svydb is predicted to be faster than the survey package if the data
set is large enough. When the data set gets larger, the increase in time for the survey
package is almost always larger than svydb. In some cases with data sets up to two
million observations, svydb is already faster.
Most of the time we can see that svydb with data sets in memory is the fastest,
followed by the survey package, and svydb with data sets in a database is always
the slowest. The reason is that it takes time for the computer to communicate with
the database, as in telling it what to do and collecting the results. However, if the
data sets are too large and cannot be fit into memory, svydb would be useful.
Some functions are just a replicate of another function but with more iterations
and more features. For example, within svydbboxplot() (Figure 4.11), svydbquantile() (Figure 4.6) is called to obtain the quantiles, svydbhist() (Figure 4.10) calls svydbmean() (Figure 4.3, Figure 4.4) to obtain the proportions of each cut, and svydbreptotal() (Figure 4.8) is computes similar to svydbtotal() (Figure 4.1, Figure 4.2)
but with more iterations with different sets of weights. Therefore, their computation
time would be similar across different sizes of data set but with some sort of a scale.
Regression (Figure 4.5) is the most interesting, though the slope of the line for
svydb.database seems to be decreasing at 1.75 observations, it is still around 1 to
1.5 seconds slower than the survey package at the two million observations mark.
Since matrix operations are not supported in a database, to get the result of a matrix multiplication is to calculate it row by row which would be slow. Therefore
svydb.database would only be expected to be faster than the survey package if
the data set is large enough, or if the model contains enough explanatory variables
which results in a unfeasible matrix operation.
Though it seems as if svydb is faster in most cases, but the structure of the survey
package is more complex, and what it can do is outside the capability of svydb, for
example, handling missing values, post-stratification and log models. It is possible
that after all the survey package would be faster if it is only computing the survey
statistics that svydb is computing.

39

Chapter 5

Usability
5.1

The package

With the help of the devtools package (Wickham, Hester, and Chang, 2018) for package building, the roxygen2 package (Wickham, Danenberg, and Eugster, 2017) for
package documentation and the formatR (Xie, 2017) package for formatting. The
full svydb package has been uploaded and can be found on Github, where it also
includes basic help pages for each functions that computes survey statistics.
The dependant packages of svydb are,
• R6 (Chang, 2017)
• Dplyr (Wickham et al., 2017)
• Rlang (Henry and Wickham, 2018)
• Survey (Lumley, 2004)
• Ggplot2 (Wickham, 2009)

5.2

Supporting functions

Along with the main functions which computes survey statistics, there are also a
number of other functions that were written to help with the implementation of the
main functions, testing those functions and to overcome difficulties where there are
inconsistency in terms of how to query between different types of databases. Details
of these functions can be found in Appendix ?.

5.3

Testing

Examples within the survey package help page was used to test svydb, all results
were identical. When examples were not available for certain functions, alternatives
were used. When data sets are too large, rounding errors will occur and is to be
expected.

41

Chapter 6

Discussion
6.1

Overview

The objective of this project is to build a tool in R to compute survey statistics for
large data sets which cannot be fit into memory and investigate whether it is feasible to do so.
From the results in chapter 4 we can see that overall, computing survey statistics
in a database is definitely feasible for certain types of statistics, that is statistics that
does not require many iterations. It is also possible that other statistics that does
not heavily depend on statistical or mathematical operations can be computed in a
database efficiently. Additionally, we have found that dplyr and dbplyr seems to be
implemented in a very efficient manner.

6.2
6.2.1

Future work
Error Messages

More user friendly error messages should be added. Currently while computing
statistics with svydb, when there is an error in the code, it will only show where the
error is without any explanations.

6.2.2

Functions

The kernel density smoother was written but it contains many bugs and is very slow,
this should be completed in a efficient manner. In general, more functions should be
built into svydb.

43

Appendix A

Codes
A.1

svydbdesign

makesvydbdesign <- R 6 Class ( " svydb . design " ,
public = list ( dataOg = NULL ,
data = NULL , dataSub = NULL ,
vars = NULL , st = NULL ,
id = NULL , wt = NULL ,
call = NULL , names = list () ,
levels = list () ,
initialize = function ( vars = NA ,
st = NA , id = NA , wt = NA , data ) {
if ( quo _ is _ null ( wt )) {
stop ( " Please provide sampling weights " )
} else {
self $ wt = as . character ( wt )[ 2 ]
}
if ( quo _ is _ null ( st )) {
data = data % >% mutate ( st = 1 )
self $ st = " st "
} else {
self $ st = as . character ( st )[ 2 ]
}
if ( quo _ is _ null ( id )) {
data = data % >% mutate ( id = row _ number ())
self $ id = " id "
} else {
self $ id = as . character ( id )[ 2 ]
}
self $ data = data % >% select ( everything ())
self $ dataOg <<- self $ data
},
setx = function ( x ) {
tc = tryCatch ( class ( x ) , error = function ( e ) e )
if ( " formula " % in % tc ) {
x = all . vars ( x )
self $ data <<- self $ data % >%

44

Appendix A. Codes
select (!! x , self $ st , self $ id , self $ wt ) % >%
filter _ all ( any _ vars (! is . na (.)))
self $ vars <<- x
} else {
x = enquo ( x )
self $ data <<- self $ data % >%
select (!! x , self $ st , self $ id , self $ wt ) % >%
filter (! is . na (!! x ))
self $ vars <<- as . character ( x )[ 2 ]
}
self $ names [[ " logged " ]] =
c ( self $ st , self $ id , self $ wt , " m _ h " )
},
addx = function ( x ) {
l = enquo ( x )
r = syms ( colnames ( self $ data ))
self $ data = self $ dataOg % >% select (!! l , !!! r )
},
getwt = function () {
self $ data % >% select ( self $ wt ) % >%
summarise _ all ( sum ) % >% pull ()
},
getmh = function () {
self $ data % >% group _ by (!! sym ( self $ st )) % >%
summarise ( m _ h = n _ distinct (!! sym ( self $ id )))
},
subset = function (... , logical = T ) {
d = self $ clone ()
if ( logical == T ) {
d $ data = d $ data % >% filter (...)
} else {
d $ data = d $ data % >% filter (!! parse _ expr (...))
}
return ( d )
},
subset _ rows = function ( from , to ) {
self $ dataSub = self $ data % >%
db _ selectRows (. , from = from , to = to )
},
storename = function ( name , obj , force = FALSE ) {
if ( force == TRUE ) {
self $ names $ logged =
self $ names $ logged [ - which (
self $ names $ logged % in % obj )]
}
if (! all ( obj % in % self $ names $ logged )) {
new = setdiff ( obj , self $ names $ logged )
self $ names [[ name ]] = c ( new )
self $ names $ logged = c ( self $ names $ logged , new )
}

A.2. svydbtotal

},
removename = function ( name , obj ) {
self $ names $ logged =
self $ names $ logged [ - which (
self $ names $ logged % in % obj )]
self $ names [[ name ]] =
( self $ names [[ name ]])[ - which (
self $ names [[ name ]] % in % obj )]
},
storelevel = function (x , name ) {
ll = list ( x )
names ( ll ) = name
self $ levels = c ( self $ levels , ll )
},
storecall = function ( x ) {
self $ call = x
},
print = function () {
nid = self $ getmh () % >% pull ( m _ h ) % >% sum ()
rows = self $ data % >% db _ nrow ()
txt = sprintf ( " svydb . design ,
% s observation ( s ) , % s Clusters \ n " ,
rows , nid )
cat ( txt )
}))
svydbdesign = function ( st = NULL , id = NULL ,
wt = NULL , data ){
st = enquo ( st )
id = enquo ( id )
wt = enquo ( wt )
d = makesvydbdesign $ new ( st = st , id = id , wt = wt ,
data = data )
d $ storecall ( match . call ())
d
}

A.2

svydbtotal

svydbtotal = function (x , num , design ,
return . total = F , ...) {
if (!( " svydb . design " % in % class ( design ))) {
stop ( " Please provide a svydb . design " )
}
if ( missing ( num )) {
stop ( " Is x a numeric or categorical variable ?
, num = T OR num = F ? " )

45

46

Appendix A. Codes
}
dsn = design $ clone ()
dsn $ setx (!! enquo ( x ))
d = dsn $ data
dsn $ storename ( " x " , colnames ( d ))
if ( num == F ) {
d = dummy _ mut (d , !! sym ( dsn $ names $ x ) , withBase = T )
}
dsn $ storename ( " x " , colnames ( d ))
d = d % >% mutate _ at ( vars ( dsn $ names $ x ) ,
funs ((. * !! sym ( dsn $ wt )))) % >%
compute ( temporary = T )
totTbl = d % >% select ( dsn $ names $ x ) % >%
summarise _ all ( sum ) % >% collect () % >% t ()
if ( return . total == TRUE ) {
colnames ( totTbl ) = " Total "
return ( totTbl )
}
varTbl = d % >% select ( dsn $ st , dsn $ id , dsn $ names $ x ) % >%
group _ by (!!! syms ( c ( dsn $ st , dsn $ id ))) % >%
summarise _ at ( vars ( dsn $ names $ x ) ,
funs ( sum (.))) % >%
compute ( temporary = T )
varTbl = inner _ join ( varTbl , dsn $ getmh () , by = dsn $ st )
barTbl = varTbl % >% select ( - one _ of ( dsn $ id )) % >%
group _ by (!! sym ( dsn $ st )) % >%
summarise _ at ( vars ( dsn $ names $ x ) ,
funs ( bar = sum (./ m _ h )))
dsn $ storename ( " bar " , colnames ( barTbl ))
varTbl = inner _ join ( varTbl , barTbl , by = dsn $ st )
‘zhi - zbar ‘ = paste ( dsn $ names $x , " -" ,
dsn $ names $ bar , collapse = " ; " )
varTbl = varTbl % >%
mutate (!!! parse _ exprs ( ‘ zhi - zbar ‘)) % >%
ungroup () % >% compute ( temporary = T )
dsn $ storename ( " diff " , colnames ( varTbl ))
varTbl = sapply ( dsn $ names $ diff , svydbVar ,
st = dsn $ st , m _ h = " m _ h " , data = varTbl )
class ( totTbl ) = " svydbstat "
attr ( totTbl , " var " ) = varTbl
attr ( totTbl , " statistic " ) <- " Total "
attr ( totTbl , " name " ) = dsn $ names $ x

A.3. svydbmean

47

return ( totTbl )
}

A.3

svydbmean

svydbmean = function (x , num , design ,
return . mean = F , ...) {
if (!( " svydb . design " % in % class ( design ))) {
stop ( " Please provide a svydb . design " )
}
if ( missing ( num )) {
stop ( " Is x a numeric or categorical variable ? ,
= T OR num = F ? " )
}
dsn = design $ clone ()
dsn $ setx (!! enquo ( x ))
d = dsn $ data
dsn $ storename ( " x " , colnames ( d ))
if ( num == F ) {
d = dummy _ mut (d , !! sym ( dsn $ names $ x ) , withBase = T )
}
dsn $ storename ( " x " , colnames ( d ))
N = dsn $ getwt ()
meanTbl = d % >%
transmute _ at ( vars ( dsn $ names $ x ) ,
funs ((. * !! sym ( dsn $ wt ))
/!! quo ( N ))) % >%
summarise _ all ( sum ) % >%
compute ( temporary = T ) % >% collect ()
if ( return . mean == TRUE ) {
colnames ( meanTbl ) = dsn $ names $ x
return ( meanTbl )
}
dhi _ exprs = paste ( dsn $ names $x , " - " , " meanTbl $ " ,
dsn $ names $x , sep = " " ,
collapse = " ; " )
varTbl = d % >% mutate ( dhi = !!! parse _ exprs ( dhi _ exprs ))
dsn $ storename ( " dhi " , colnames ( varTbl ))
varTbl = varTbl % >%
mutate _ at ( vars ( dsn $ names $ dhi ) ,
funs ((. * !! sym ( dsn $ wt ))/!! quo ( N ))) % >%
select ( dsn $ st , dsn $ id ,

48

Appendix A. Codes
dsn $ names $ dhi )
barTbl = varTbl % >% select ( dsn $ st , dsn $ names $ dhi ) % >%
group _ by (!! sym ( dsn $ st )) % >% summarise _ all ( sum )
barTbl = inner _ join ( barTbl , dsn $ getmh () ,
by = dsn $ st ) % >%
mutate _ at ( vars ( dsn $ names $ dhi ) ,
funs ( bar = ./ m _ h )) % >%
select ( - one _ of ( dsn $ names $ dhi ))
dsn $ storename ( " bar " , colnames ( barTbl ))
varTbl = varTbl % >%
group _ by (!!! syms ( c ( dsn $ st , dsn $ id ))) % >%
summarise _ all ( sum ) % >% compute ( temporary = T )
varTbl = inner _ join ( varTbl , barTbl , by = dsn $ st )
‘dhi - dbar ‘ = paste ( " ‘" , dsn $ names $ dhi , " ‘" , " -" ,
" ‘" , dsn $ names $ bar , " ‘" ,
collapse = " ; " , sep = " " )
varTbl = varTbl % >%
mutate (!!! parse _ exprs ( ‘ dhi - dbar ‘)) % >%
ungroup () % >% compute ( temporary = T )
dsn $ storename ( " diff " , colnames ( varTbl ))
varTbl = sapply ( dsn $ names $ diff , svydbVar ,
st = dsn $ st , m _ h = " m _ h " , data = varTbl )
meanTbl = t ( meanTbl )
class ( meanTbl ) = " svydbstat "
attr ( meanTbl , " var " ) = varTbl
attr ( meanTbl , " statistic " ) <- " Mean "
attr ( meanTbl , " name " ) = dsn $ names $ x
return ( meanTbl )

}

A.4

svydblm

svydblm = function ( formula , design ) {
if (!( " svydb . design " % in % class ( design ))) {
stop ( " Please provide a svydb . design " )
}
if ( missing ( formula )) {
stop ( " Please provide a formula " )
}
fit = list ()
fit $ call = match . call ()
dsn = design $ clone ()
dsn $ setx ( formula )

A.4. svydblm

49

d = dsn $ data
dsn $ storename ( " y " , all . vars ( formula )[ 1 ])
dsn $ storename ( " variables " , all . vars ( formula )[ - 1 ])
d = d % >% mutate ( ‘:= ‘(!! sym ( dsn $ wt ) ,
case _ when ( is . na (!! sym ( dsn $ names $ y )) ~
0 , TRUE ~ !! sym ( dsn $ wt ))))
d = d % >% mutate ( ‘:= ‘(!! sym ( dsn $ names $ y ) ,
case _ when ( is . na (!! sym ( dsn $ names $ y )) ~
0 , TRUE ~ !! sym ( dsn $ names $ y ))))
d = d % >% filter _ all ( all _ vars (! is . na (.)))
d = d % >% mutate ( intercept = 1 )
dsn $ storename ( " intercept " , colnames ( d ))
xy = d
facVar = attr ( terms ( formula , specials = " factor " ) ,
" specials " )$ factor
if (! is . null ( facVar )) {
facVar = facVar - 1
dsn $ storename ( " factor " , dsn $ names $ variables [ facVar ] ,
force = T )
for ( i in 1 : length ( facVar )) {
dd = dummy _ mut ( xy , by =
!! sym ( dsn $ names $ factor [ i ]) ,
withBase = F , return . level = T )
xy = dd $ dum
dsn $ storelevel ( x = dd $ levels ,
name = dsn $ names $ factor [ i ])
dsn $ names $ variables = c ( dsn $ names $ variables ,
dsn $ names $ factor [ i ])
dsn $ names $ variables =
dsn $ names $ variables [ -( grep (
dsn $ names $ factor [ i ] ,
dsn $ names $ variables )[ 1 ])]
}
}
fit $ terms = terms ( paste ( " ~ " , paste ( dsn $ names $ variables ,
collapse = " + " )) % >% as . formula )
dsn $ storename ( " dummy " , colnames ( xy ))
xy = compute ( xy )
dsn $ storename ( " variables " , c ( dsn $ names $ variables ,
dsn $ names $ dummy ) , force = T )
if (! is . null ( facVar )) {
dsn $ removename ( " variables " , dsn $ names $ factor )
}
db _ xtwx _ i = function (x , col , wt , data ) {
data = data % >% summarise _ at ( vars ( x ) ,
funs ( xtx = sum (. * (!! sym ( col )) *
(!! sym ( wt )))))

50

Appendix A. Codes
return ( data )
}
xtx = lapply ( c ( dsn $ names $ intercept ,
dsn $ names $ variables ) , db _ xtwx _i ,
x = c ( dsn $ names $ intercept ,
dsn $ names $ variables ) , wt = dsn $ wt ,
data = xy )
xtx = Reduce ( full _ join , xtx ) % >% collect () % >%
as . matrix ()
colnames ( xtx ) =
c ( dsn $ names $ intercept , dsn $ names $ variables )
xty = xy % >% transmute _ at ( vars ( dsn $ names $ intercept ,
dsn $ names $ variables ) , funs (. * (!! sym ( dsn $ wt )) *
(!! sym ( dsn $ names $ y )))) % >% summarise _ all ( sum ) % >%
collect () % >% t ()
xy = xy % >% filter (!(!! sym ( dsn $ wt )) == 0 ) % >%
compute ()
beta = solve ( xtx ) %*% xty
fit $ coefname =
c ( dsn $ names $ intercept , dsn $ names $ variables )
fit $ coefficients = beta % >% t ()

dsn $ storename ( " xb " , paste ( dsn $ names $ variables ,
" _ xb " , sep = " " ))
e = paste ( dsn $ names $y , " - " , beta [ " intercept " ,
] , " - " , paste ( rownames ( beta )[ - 1 ] , " * " ,
beta [ - 1 , ] , sep = " " , collapse = " - " ) ,
sep = " " )
xy = xy % >% mutate ( residuals = !! parse _ expr ( e )) % >%
select ( - one _ of ( dsn $ names $ y ))
dsn $ storename ( " residuals " , colnames ( xy ))
res = xy % >% select ( residuals )
fit $ residuals = res
varTbl = xy % >% mutate _ at ( vars ( dsn $ names $ intercept ,
dsn $ names $ variables ) ,
funs (. * (!! sym ( dsn $ names $ residuals )) *
(!! sym ( dsn $ wt )))) % >% group _ by (!! sym ( dsn $ st ) ,
!! sym ( dsn $ id )) % >% summarise _ all ( sum ) % >%
compute ()
dsn $ storename ( " zhi " , c ( dsn $ names $ intercept ,
dsn $ names $ variables ) , force = T )
barTbl = varTbl % >% select ( dsn $ st , dsn $ names $ zhi ) % >%
group _ by (!! sym ( dsn $ st )) % >% summarise _ all ( sum )
mh = dsn $ getmh ()
barTbl = inner _ join ( barTbl , mh , by = dsn $ st ) % >%

A.5. svydbquantile

51

mutate _ at ( vars ( dsn $ names $ intercept ,
dsn $ names $ variables ) , funs ( bar = ./ m _ h )) % >%
compute ()
dsn $ storename ( " bar " , colnames ( barTbl ))
varTbl = inner _ join ( varTbl % >% select ( dsn $ st ,
dsn $ names $ zhi ) , barTbl % >% select ( dsn $ st ,
dsn $ names $ bar , m _ h ) , by = dsn $ st )
‘zhi - zbar ‘ = paste ( " ‘" , dsn $ names $ zhi , " ‘" ,
" -" , " ‘" , dsn $ names $ bar , " ‘" , collapse = " ; " ,
sep = " " )
varTbl = varTbl % >%
mutate (!!! parse _ exprs ( ‘ zhi - zbar ‘)) % >%
ungroup () % >% compute ()
dsn $ storename ( " diff " , colnames ( varTbl ))
covDiag = sapply ( dsn $ names $ diff , svydbVar ,
st = dsn $ st , m _ h = " m _ h " , data = varTbl )
e = outer ( dsn $ names $ diff , dsn $ names $ diff ,
paste , sep = " ," )
e = e [ lower . tri ( e )] % >% strsplit (. , " ," )
covLt = sapply (e , svydbVar 2 , st = dsn $ st ,
m _ h = " m _ h " , data = varTbl )
cov . mat = diag ( covDiag )
cov . mat [ lower . tri ( cov . mat )] = covLt
cov . mat [ upper . tri ( cov . mat )] =
t ( cov . mat )[ upper . tri ( cov . mat )]
fit $ cov . unscaled = solve ( xtx ) %*% cov . mat %*%
solve ( xtx )
colnames ( fit $ cov . unscaled ) = rownames ( fit $ cov . unscaled )
fit $ formula = formula
fit $ df . residual = sum ( mh % >% select ( m _ h ) % >%
pull ) - db _ nrow ( mh ) - nrow ( beta ) + 1
fit $ call = match . call ()
fit $ design = dsn
class ( fit ) = c ( " svydblm " , " lm " )
return ( fit )
}

A.5

svydbquantile

svydbquantile = function (x , quantiles = 0 . 5 , design ) {
oldoptions = options ( " survey . lonely . psu " )
options ( survey . lonely . psu = " adjust " )
dsn = design $ clone ()
dsn $ setx (!! enquo ( x ))
d = dsn $ data

52

Appendix A. Codes
dsn $ storename ( " x " , colnames ( d ))
q _ name = quantiles
qvec = c ()
b = T
for ( i in 1 : length ( quantiles )) {
if ( quantiles [ i ] >= 1 ) {
qvec [ i ] =
dbmax ( data = d , var = !! sym ( dsn $ names $ x ))
} else if ( quantiles [ i ] <= 0 ) {
qvec [ i ] =
dbmin ( data = d , var = !! sym ( dsn $ names $ x ))
} else {
if ( b ) {
N = dsn $ getwt ()
nrowdata = db _ nrow ( d )
sampN = ceiling ( nrowdata ^( 2 / 3 ))
d = compute ( d )
if (! is . null ( d $ src $ con )) {
sdata = d % >%
svydb _ monet _ sampleN ( sampN ) % >%
tbl _ df ()
} else {
sdata = d % >% sample _ n ( sampN ) % >%
tbl _ df ()
}
sdata =
sdata % >% rename ( x = !! sym ( dsn $ names $ x ))
s . surv = svydesign ( id = c 2 f ( dsn $ id ) ,
st = c 2 f ( dsn $ st ) , weights = c 2 f ( dsn $ wt ) ,
data = sdata , nest = T )
b = F
}
notfound = TRUE
while ( notfound ) {
q = svyquantile (~ x , s . surv , quantiles [ i ] ,
alpha = 0 . 1 , ci = TRUE , na . rm = T )
temp _ lq = q $ CIs [ 1 ]
temp _ uq = q $ CIs [ 2 ]
readIn = d % >% select ( x = dsn $ names $x ,
wt = dsn $ wt ) % >% filter ( x >= temp _ lq &
x <= temp _ uq )
readIn _ wts = readIn % >% select ( wt ) % >%
summarise ( sum ( wt )) % >% pull ()
notRead = d % >% select ( x = dsn $ names $x ,

A.6. svydbtable

53
wt = dsn $ wt ) % >% filter ( x < temp _ lq )
notRead _ wts = notRead % >% select ( wt ) % >%
summarise ( sum ( wt )) % >% pull ()
thold = ( N * quantiles [ i ])
if (( notRead _ wts <= thold ) & ( thold <
( readIn _ wts + notRead _ wts ))) {
qs = readIn % >% arrange ( x )
qs = qs % >% collect () % >%
mutate ( wt 2 = cumsum ( wt ))
c _ wts = N * quantiles [ i ] - notRead _ wts
qs = qs % >%
filter ( wt 2 >= !! quo ( c _ wts )) % >%
select ( x ) % >% slice ( 1 ) % >% pull ()
qvec [ i ] = qs
notfound = F
} else {
if (! is . null ( d $ src $ con )) {
sdata = d % >%
svydb _ monet _ sampleN ( sampN ) % >%
tbl _ df () % >%
rename ( x = !! sym ( dsn $ names $ x ))
} else {
sdata = d % >% sample _ n ( sampN ) % >%
tbl _ df () % >%
rename ( x = !! sym ( dsn $ names $ x ))
}
s . surv = svydesign ( id = c 2 f ( dsn $ id ) ,
st = c 2 f ( dsn $ st ) , weights = c 2 f ( dsn $ wt ) ,
data = sdata , nest = T )
}
}

}
}
names ( qvec ) = as . character ( q _ name )
options ( oldoptions )
return ( qvec )
}

A.6

svydbtable

svydbtable = function ( formula , design , as . local = F ) {
dsn = design $ clone ()
dsn $ setx ( formula )
d = dsn $ data
d = d % >% filter _ all ( all _ vars (! is . na (.)))
dsn $ storename ( " all " , colnames ( d ))
ff = all . vars ( formula )
dsn $ storename ( " base " , ff [ 1 ] , force = T )

54

Appendix A. Codes

if ( length ( ff ) == 1 ) {
out = d % >% group _ by (!! sym ( dsn $ names $ base )) % >%
summarise ( wt = sum (!! sym ( dsn $ wt ))) % >%
arrange (!! sym ( dsn $ names $ base ))
if ( as . local == T ) {
out = collect ( out )
}
return ( out )
}
dsn $ storename ( " by " , ff [ 2 ] , force = T )
d = d % >% select ( dsn $ names $ all , dsn $ wt ) % >%
dummy _ mut ( data = . , by = !! sym ( dsn $ names $ by ) ,
withBase = T )
dsn $ storename ( " dummy " , colnames ( d ))
d = d % >% select ( - one _ of ( dsn $ names $ by ))
d = compute ( d )
if ( length ( ff ) == 2 ) {
out = d % >% group _ by (!! sym ( dsn $ names $ base )) % >%
summarise _ at ( vars ( dsn $ names $ dummy ) ,
funs ( sum (. * (!! sym ( dsn $ wt ))))) % >%
arrange (!! sym ( dsn $ names $ base ))
if ( as . local == T ) {
out = collect ( out )
}
return ( out )
}
dsn $ storename ( " others " , ff [ - c ( 1 , 2 )] , force = T )
combnTbl = d % >% select ( dsn $ names $ others ) % >%
distinct () % >%
arrange (!!! syms ( dsn $ names $ others )) % >%
collect ()
combnLst = split ( combnTbl , seq ( 1 , nrow ( combnTbl )))
sTbls = function ( by ) {
nname = paste ( colnames ( by ) , " = " , by ,
collapse = " & " )
con = gsub ( pattern = " = " , replacement = " == " ,
nname )
d = d % >% filter (!! parse _ expr ( con )) % >%
select ( - one _ of ( colnames ( by ))) % >%
group _ by (!! sym ( dsn $ names $ base )) % >%
summarise _ at ( vars ( dsn $ names $ dummy ) ,
funs ( sum (. * (!! sym ( dsn $ wt ))))) % >%
arrange (!! sym ( dsn $ names $ base )) % >%
list (.)
names ( d ) = nname
d

A.7. svydbby
}
out = lapply ( combnLst , sTbls ) % >% flatten ()
if ( as . local == T ) {
out = lapply ( out , collect )
}
return ( out )
}

A.7

svydbby

svydbtable = function ( formula , design , as . local = F ) {
dsn = design $ clone ()
dsn $ setx ( formula )
d = dsn $ data
d = d % >% filter _ all ( all _ vars (! is . na (.)))
dsn $ storename ( " all " , colnames ( d ))
ff = all . vars ( formula )
dsn $ storename ( " base " , ff [ 1 ] , force = T )
if ( length ( ff ) == 1 ) {
out = d % >% group _ by (!! sym ( dsn $ names $ base )) % >%
summarise ( wt = sum (!! sym ( dsn $ wt ))) % >%
arrange (!! sym ( dsn $ names $ base ))
if ( as . local == T ) {
out = collect ( out )
}
return ( out )
}
dsn $ storename ( " by " , ff [ 2 ] , force = T )
d = d % >% select ( dsn $ names $ all , dsn $ wt ) % >%
dummy _ mut ( data = . , by = !! sym ( dsn $ names $ by ) ,
withBase = T )
dsn $ storename ( " dummy " , colnames ( d ))
d = d % >% select ( - one _ of ( dsn $ names $ by ))
d = compute ( d )
if ( length ( ff ) == 2 ) {
out = d % >% group _ by (!! sym ( dsn $ names $ base )) % >%
summarise _ at ( vars ( dsn $ names $ dummy ) ,
funs ( sum (. * (!! sym ( dsn $ wt ))))) % >%
arrange (!! sym ( dsn $ names $ base ))
if ( as . local == T ) {
out = collect ( out )
}
return ( out )
}

55

56

Appendix A. Codes

dsn $ storename ( " others " , ff [ - c ( 1 , 2 )] , force = T )
combnTbl = d % >% select ( dsn $ names $ others ) % >%
distinct () % >%
arrange (!!! syms ( dsn $ names $ others )) % >%
collect ()
combnLst = split ( combnTbl , seq ( 1 , nrow ( combnTbl )))
sTbls = function ( by ) {
nname = paste ( colnames ( by ) , " = " , by ,
collapse = " & " )
con = gsub ( pattern = " = " , replacement = " == " ,
nname )
d = d % >% filter (!! parse _ expr ( con )) % >%
select ( - one _ of ( colnames ( by ))) % >%
group _ by (!! sym ( dsn $ names $ base )) % >%
summarise _ at ( vars ( dsn $ names $ dummy ) ,
funs ( sum (. * (!! sym ( dsn $ wt ))))) % >%
arrange (!! sym ( dsn $ names $ base )) % >%
list (.)
names ( d ) = nname
d
}
out = lapply ( combnLst , sTbls ) % >% flatten ()
if ( as . local == T ) {
out = lapply ( out , collect )
}
return ( out )
}

A.8

svydbrepdesign

makesvydbrepdesign <- R 6 Class ( " svydb . repdesign " ,
public = list ( dataOg = NULL , data = NULL ,
vars = NULL , st = NULL , id = NULL , wt = NULL ,
repwt = NULL , scale = NULL , names = list () ,
initialize = function ( vars = NA , st = NA ,
id = NA , wt = NA , repwt = NULL , scale ,
data ) {
if ( quo _ is _ null ( wt )) {
stop ( " Please provide sampling weights " )
} else {
self $ wt = as . character ( wt )[ 2 ]
}
if ( is . null ( repwt )) {
stop ( " Please provide replicate weights " )
} else {

A.8. svydbrepdesign

57
self $ repwt = grep ( pattern = repwt ,
colnames ( data ) , value = T )

}
if ( quo _ is _ null ( st )) {
data = data % >% mutate ( st = 1 )
self $ st = " st "
} else {
self $ st = as . character ( st )[ 2 ]
}
if ( quo _ is _ null ( id )) {
data = data % >% mutate ( id = row _ number ())
self $ id = " id "
} else {
self $ id = as . character ( id )[ 2 ]
}
self $ scale = scale
self $ data = data % >% select ( everything ())
self $ dataOg <<- self $ data
} , setx = function ( x ) {
tc = tryCatch ( class ( x ) , error = function ( e ) e )
if ( " formula " % in % tc ) {
x = all . vars ( x )
self $ data <<- self $ data % >% select (!! x ,
st = self $ st , id = self $ id ,
self $ wt , self $ repwt ) % >%
filter _ all ( any _ vars (! is . na (.)))
self $ vars <<- x
} else {
x = enquo ( x )
self $ data <<- self $ data % >% select (!! x ,
st = self $ st , id = self $ id ,
self $ wt , self $ repwt ) % >%
filter (! is . na (!! x ))
self $ vars <<- as . character ( x )[ 2 ]
}
self $ names [[ " logged " ]] = c ( self $ st ,
self $ id , self $ wt , self $ repwt ,
"m_h")
} , addx = function ( x ) {
l = enquo ( x )
r = syms ( colnames ( self $ data ))
self $ data = self $ dataOg % >% select (!! l ,
!!! r )
} , getwt = function () {
self $ data % >% select ( self $ wt ) % >%
summarise _ all ( sum ) % >% pull ()
} , getmh = function () {
self $ data % >% group _ by (!! sym ( self $ st )) % >%

58

Appendix A. Codes
summarise ( m _ h = n _ distinct (!! sym ( self $ id )))
} , subset = function (... , logical = T ) {
d = self $ clone ()
if ( logical == T ) {
d $ data = d $ data % >% filter (...)
} else {
d $ data = d $ data % >%
filter (!! parse _ expr (...))
}
return ( d )
} , subset _ rows = function ( from , to ) {
self $ dataSub = self $ data % >% db _ selectRows (. ,
from = from , to = to )
} , storename = function ( name , obj , force = FALSE ) {
if ( force == TRUE ) {
self $ names $ logged =
self $ names $ logged [ - which (
self $ names $ logged % in % obj )]
}
if (! all ( obj % in % self $ names $ logged )) {
new = setdiff ( obj , self $ names $ logged )
self $ names [[ name ]] = c ( new )
self $ names $ logged = c ( self $ names $ logged ,
new )
}
} , removename = function ( name , obj ) {
self $ names $ logged =
self $ names $ logged [ - which (
self $ names $ logged % in % obj )]
self $ names [[ name ]] =
( self $ names [[ name ]])[ - which (
self $ names [[ name ]] % in % obj )]
} , print = function () {
rows = self $ data % >% db _ nrow ()
txt = sprintf ( " svydb . repdesign ,
% s observation ( s ) , % s sets of
replicate weights , scale = % s " ,
rows , length ( self $ repwt ) , self $ scale )
cat ( txt )

}))
svydbrepdesign = function ( st = NULL , id = NULL ,
wt = NULL , repwt = NULL , scale , data ) {
st = enquo ( st )
id = enquo ( id )
wt = enquo ( wt )
d = makesvydbrepdesign $ new ( st = st , id = id ,
wt = wt , repwt = repwt , scale = scale ,
data = data )
d

A.9. svydbreptotal

59

}

A.9

svydbreptotal

svydbreptotal = function (x , design ,
num , return . replicates = F ) {
x = enquo ( x )
if (!( " svydb . repdesign " % in % class ( design ))) {
stop ( " Please provide a svydb . repdesign " )
}
if ( missing ( num )) {
stop ( " Is x a numeric or categorical variable ? ,
num = T OR num = F ? " )
}
dsn = design $ clone ()
dsn $ setx (!! enquo ( x ))
d = dsn $ data
dsn $ storename ( " x " , colnames ( d ))
if ( num == F ) {
d = dummy _ mut (d , !! sym ( dsn $ names $ x ) , withBase = T )
}
dsn $ storename ( " x " , colnames ( d ))
fullTotTbl = d % >% summarise _ at ( vars ( dsn $ names $ x ) ,
funs ( sum (. * (!! sym ( dsn $ wt ))))) % >% collect ()
cnt = 1
getRepTots = function ( names , fullTot ) {
replicates = d % >% summarise _ at ( vars ( dsn $ repwt ) ,
funs ( sum ((. * !! sym ( names ))))) % >%
collect ()
repTot = replicates % >% summarise _ all ( funs ((. !! quo ( fullTot [ cnt ]))^ 2 ))
cnt <<- cnt + 1
if ( return . replicates == T ) {
list ( replicates = replicates ,
repVar = db _ rowSums ( repTot ) % >%
transmute _ all (
funs (. * !! quo ( dsn $ scale ))) % >%
collect ())
} else {
list ( repVar = db _ rowSums ( repTot ) % >%
transmute _ all ( funs (. *
!! quo ( dsn $ scale ))) % >%

60

Appendix A. Codes
collect ())
}
}
ans = lapply ( colnames ( fullTotTbl ) , getRepTots ,
fullTot = as . vector ( t ( fullTotTbl )))
repVar = lapply ( ans , function ( x ) x $ repVar ) % >%
Reduce ( rbind , .) % >% pull ()
tot = fullTotTbl % >% t () % >% as . vector ()
attr ( tot , " var " ) = repVar
attr ( tot , " statistic " ) <- " Total "
attr ( tot , " name " ) = dsn $ names $ x
if ( return . replicates == T ) {
replicates =
lapply ( ans , function ( x ) x $ replicates % >%
collect ()) % >% Reduce ( rbind , .)
tot = list ( svydbrepstat = tot ,
replicates = replicates )
}
class ( tot ) = c ( " svydbrepstat " )
return ( tot )

}

A.10

svydbrepmean

svydbrepmean = function (x , design , num ,
return . replicates = F ) {
x = enquo ( x )
if (!( " svydb . repdesign " % in % class ( design ))) {
stop ( " Please provide a svydb . repdesign " )
}
if ( missing ( num )) {
stop ( " Is x a numeric or categorical variable ? ,
num = T OR num = F ? " )
}
dsn = design $ clone ()
dsn $ setx (!! enquo ( x ))
d = dsn $ data
dsn $ storename ( " x " , colnames ( d ))
if ( num == F ) {
d = dummy _ mut (d , !! sym ( dsn $ names $ x ) , withBase = T )
}

A.10. svydbrepmean
dsn $ storename ( " x " , colnames ( d ))
N = dsn $ getwt ()
fullMeanTbl = d % >% summarise _ at ( vars ( dsn $ names $ x ) ,
funs ( sum (. * (!! sym ( dsn $ wt )))/!! quo ( N ))) % >%
collect ()
repN = d % >% select ( dsn $ repwt ) % >%
summarise _ all ( sum ) % >%
collect ()
repN = paste ( colnames ( repN ) , " / " , repN ,
collapse = " ; " )
cnt = 1
getRepTots = function ( names , fullMean ) {
replicates = d % >% summarise _ at ( vars ( dsn $ repwt ) ,
funs ( sum (. * !! sym ( names )))) % >%
transmute (!!! parse _ exprs ( repN )) % >%
compute ()
repMean = replicates % >% summarise _ all ( funs ((. !! quo ( fullMean [ cnt ]))^ 2 ))
cnt <<- cnt + 1
if ( return . replicates == T ) {
list ( replicates = replicates ,
repVar = db _ rowSums ( repMean ) % >%
transmute _ all ( funs (. *
!! quo ( dsn $ scale ))) % >%
collect ())
} else {
list ( repVar = db _ rowSums ( repMean ) % >%
transmute _ all ( funs (. *
!! quo ( dsn $ scale ))) % >%
collect ())
}
}
ans = lapply ( colnames ( fullMeanTbl ) , getRepTots ,
fullMean = as . vector ( t ( fullMeanTbl )))
repVar = lapply ( ans , function ( x ) x $ repVar ) % >%
Reduce ( rbind , .) % >% pull ()
means = fullMeanTbl % >% t () % >% as . vector ()
attr ( means , " var " ) = repVar
attr ( means , " statistic " ) <- " Mean "
attr ( means , " name " ) = dsn $ names $ x
if ( return . replicates == T ) {
replicates = lapply ( ans , function ( x )
x $ replicates % >%
collect ()) % >% Reduce ( rbind , .)
means = list ( svydbrepstat = means ,
replicates = replicates )
}

61

62

Appendix A. Codes
class ( means ) = c ( " svydbrepstat " )
return ( means )

}

A.11

svydbhist

svydbhist = function (x , design , binwidth = NULL ,
xlab = " x " , ylab = " Density " , ...) {
if (!( " svydb . design " % in % class ( design ))) {
stop ( " Please provide a svydb . design " )
}
dsn = design $ clone ()
dsn $ setx (!! enquo ( x ))
d = dsn $ data
dsn $ storename ( " x " , colnames ( d ))
d _ n = d % >% db _ nrow ()
x _ max = d % >% dbmax (!! sym ( dsn $ names $ x ) , asNum = T )
x _ min = d % >% dbmin (!! sym ( dsn $ names $ x ) , asNum = T )
x _ range = x _ max - x _ min
if ( is . null ( binwidth )) {
binwidth = ceiling ( log 2 ( d _ n ) + 1 )
pbreaks = pretty ( c ( x _ min , x _ max ) , n = binwidth ,
min . n = 1 )
} else {
pbreaks = seq ( from = floor ( x _ min ) ,
to = ceiling ( x _ max ) ,
by = binwidth )
}
d = db _ cut 2 ( var = !! sym ( dsn $ names $ x ) , breaks = pbreaks ,
data = d ) % >% arrange ( cut )
dsn $ data = d
props = svydbmean ( x = cut , design = dsn , num = F ,
return . mean = T ) % >% collect () % >% t ()
colnames ( props ) = " Mean "
mids = pbreaks [ - length ( pbreaks )] + diff ( pbreaks )/ 2
if ( length ( mids ) != nrow ( props )) {
props = tbl _ df ( props ) % >%
mutate ( uqid = row _ number ())
d = left _ join ( mids % >% tbl _ df () % >%
mutate ( uqid = row _ number ()) ,
props , by = " uqid " )
d = d % >% mutate ( Mean = case _ when ( is . na ( Mean ) ~
0 , TRUE ~ Mean )) % >% select ( - uqid )

A.12. svydbboxplot
} else {
d = cbind ( mids , props ) % >% tbl _ df ()
}
colnames ( d ) = c ( " x " , " y " )
d $ y = d $ y / diff ( pbreaks )
p = ggplot ( d ) + geom _ col ( aes (x , y )) +
labs ( x = dsn $ names $x , y = ylab )
print ( p )
invisible ( p )
}

A.12

svydbboxplot

svydbboxplot = function (x , groups = NULL , design ,
varwidth = F , outlier = F , all . outlier = F ) {
groups = enquo ( groups )
dsn = design $ clone ()
dsn $ setx (!! enquo ( x ))
d = dsn $ data
dsn $ storename ( " x " , colnames ( d ))
if ( quo _ is _ null ( groups )) {
boxes = svydbquantile ( x = !! sym ( dsn $ names $ x ) ,
quantile = c ( 0 , 0 . 2 5 , 0 . 5 , 0 . 7 5 , 1 ) ,
design = dsn ) % >% t () % >% tbl _ df () % >%
mutate ( " " )
ax = c ( x = " " , y = dsn $ names $ x )
colnames ( boxes ) = c ( as . character ( letters [ 1 : 5 ]) ,
"x")
} else {
group _ name = as . character ( groups )[ 2 ]
dsn $ addx ( group _ name )
d = dsn $ data
dsn $ storename ( " groups " , colnames ( d ))
group _ levels = distinct (d ,
!! sym ( dsn $ names $ groups )) % >%
collect ()
group _ names = paste ( colnames ( group _ levels ) ,
pull ( group _ levels ) , sep = " " )
group _ names 2 = paste ( colnames ( group _ levels ) ,
paste ( " ’" , pull ( group _ levels ) , " ’" ,
sep = " " ) , sep = " " )
f = function ( x ) {
svydbquantile ( x = !! sym ( dsn $ names $ x ) ,
quantile = c ( 0 , 0 . 2 5 , 0 . 5 , 0 . 7 5 ,

63

64

Appendix A. Codes
1 ) , design = dsn $ subset (x , logical = F ))
}
boxes = sapply ( gsub ( pattern = " " ,
replacement = " == " , x = ,
group _ names 2 ) , f )
boxes = t ( boxes ) % >% tbl _ df () % >%
bind _ cols ( group _ levels )
ax = c ( x = dsn $ names $ groups , y = dsn $ names $ x )
colnames ( boxes ) = c ( as . character ( letters [ 1 : 5 ]) ,
"x")
boxes $ x = as . character ( boxes $ x )
}
haveOut = F
if ( outlier == T ) {
boxes = boxes % >% mutate ( outUP = d + 1 . 5 *
( d - b ) , checkUP = ifelse ( outUP <
e , T , F ) , outLow = b - 1 . 5 * ( d b ) , checkLow = ifelse ( outLow > a ,
T , F ))
if ( any ( boxes $ checkLow ) == T ) {
haveOut = T
boxes = boxes % >% mutate ( a = ifelse ( checkLow ==
T , outLow , a ))
outls = paste ( gsub ( pattern = " " ,
replacement = " == " , x = group _ names 2 ) ,
" & " , dsn $ names $x , " <" , boxes $a ,
collapse = " | " )
outlsLow = d % >%
filter (!!! parse _ exprs ( outls )) % >%
select ( x = dsn $ names $ groups ,
y = dsn $ names $ x ) % >%
mutate ( x = as . character ( x ))
if ( all . outlier == F ) {
outlsLow = outlsLow % >% group _ by ( x ) % >%
summarise ( y = min ( y ))
}
outlsLow = outlsLow % >% tbl _ df ()
}
if ( any ( boxes $ checkUP ) == T ) {
haveOut = T
boxes = boxes % >% mutate ( e = ifelse ( checkUP ==
T , outUP , e ))
outls = paste ( gsub ( pattern = " " ,
replacement = " == " , x = group _ names 2 ) ,
" & " , dsn $ names $x , " >" , boxes $e ,
collapse = " | " )
outlsUP = d % >%
filter (!!! parse _ exprs ( outls )) % >%

A.13. svydbhexbin, svydbhexplot
select ( x = dsn $ names $ groups ,
y = dsn $ names $ x ) % >%
mutate ( x = as . character ( x ))
if ( all . outlier == F ) {
outlsUP = outlsUP % >% group _ by ( x ) % >%
summarise ( y = max ( y ))
}
outlsUP = outlsUP % >% tbl _ df ()
}
outls = bind _ rows ( outlsUP , outlsLow )
}
p = ggplot ( boxes ) + labs ( x = ax [ " x " ] , y = ax [ " y " ])
if ( varwidth == T ) {
boxwid = d % >%
group _ by (!! sym ( dsn $ names $ groups )) % >%
summarise ( wid = n ()) % >% collect ()
p $ data = p $ data % >%
mutate ( width = ( boxwid $ wid / sum ( boxwid $ wid )))
p = p + geom _ boxplot ( aes ( x = as . factor ( x ) ,
ymin = a , lower = b , middle = c , upper = d ,
ymax = e , width = width ) , stat = " identity " )
} else {
p = p + geom _ boxplot ( aes ( x = as . factor ( x ) ,
ymin = a , lower = b , middle = c , upper = d ,
ymax = e ) , stat = " identity " )
}
if ( haveOut == T ) {
p = p + geom _ point ( data = outls , aes ( x = x ,
y = y ))
}
print ( p )
return ( p )
}

A.13

svydbhexbin, svydbhexplot

svydbhcell 2 xy = function ( d ) {
xbins = d $ xbins
xbnds = d $ xbnds
c 3 = diff ( xbnds )/ xbins
ybnds = d $ ybnds
c 4 = ( diff ( ybnds ) * sqrt ( 3 ))/( 2 * d $ shape *
xbins )
jmax = d $ dimen [ 2 ]
cell = d $ cell - 1

65

66

Appendix A. Codes
i
j
y
x

=
=
=
=

cell %/% jmax
cell %% jmax
c 4 * i + ybnds [ 1 ]
c 3 * ifelse ( i %% 2 == 0 , j , j + 0 . 5 ) + xbnds [ 1 ]

return ( list ( x = x , y = y ))
}

svydbhbin = function ( xy , x , y , xName , yName , cell ,
cnt , xcm , ycm , size , shape , rx , ry , bnd , n ) {
xmin = rx [ 1 ]
ymin = ry [ 1 ]
xr = rx [ 2 ] - xmin
yr = ry [ 2 ] - ymin
c 1 = size / xr
c 2 = size * shape /( yr * sqrt ( 3 ))
jinc = floor ( bnd [ 2 ])
lat = floor ( jinc + 1 )
iinc = floor ( 2 * jinc )
lmax = floor ( bnd [ 1 ] * as . integer ( bnd [ 2 ]))
con 1 = 0 . 2 5
con 2 = 1 / 3
xy = xy % >% mutate ( sx = !! quo ( c 1 ) , sy = !! quo ( c 2 ) ,
xmin = !! quo ( xmin ) , ymin = !! quo ( ymin ))
xy = xy % >% mutate ( sx = sx * ( x - xmin ))
xy = xy % >% mutate ( sy = sy * ( y - ymin ))
xy = xy % >% mutate ( j 1 = floor ( sx + 0 . 5 ) ,
i 1 = floor ( sy + 0 . 5 ))
xy = xy % >% mutate ( dist 1 = ( sx - j 1 )^ 2 + 3 *
( sy - i 1 )^ 2 , iinc = !! quo ( iinc ) , lat = !! quo ( lat ))
xy = xy % >% mutate ( con 1 = !! quo ( con 1 ) ,
con 2 = !! quo ( con 2 ) , j 2 = floor ( sx ) ,
i 2 = floor ( sy ))
xy = xy % >% mutate ( con 3 = ( sx - j 2 - 0 . 5 )^ 2 +
3 * ( sy - i 2 - 0 . 5 )^ 2 )
xy = xy % >% mutate ( L = case _ when ( dist 1 < con 1 ~
floor ( i 1 * iinc + j 1 + 1 ) , dist 1 > con 2 ~
floor ( floor ( sy ) * iinc + floor ( as . double ( sx )) +
lat ) , TRUE ~ case _ when ( dist 1 <= con 3 ~
floor ( i 1 * iinc + j 1 + 1 ) , TRUE ~ floor ( i 2 *
iinc + j 2 + lat ))))
Lfulltbl = xy % >% select (x , y , L )
cmsTbl = Lfulltbl % >% group _ by ( L ) % >%
summarise ( xcm = mean ( x ) , ycm = mean ( y )) % >%
arrange ( L ) % >% select ( - L ) % >%
collect ()
Ltbl = xy % >% select ( L 2 = L , wt ) % >% group _ by ( L 2 ) % >%

A.13. svydbhexbin, svydbhexplot

67

summarise ( cnt = sum ( wt ))
xy = xy % >% select (x , y ) % >% mutate ( L 2 = row _ number ())
xy = left _ join ( xy , Ltbl , by = " L 2 " ) % >%
mutate ( cnt = case _ when ( is . na ( cnt ) ~
0 , TRUE ~ cnt )) % >%
arrange ( L 2 )
cntsTbl = xy % >% rename ( cell = L 2 ) % >%
mutate ( lt 1 = case _ when ( cnt > 0 ~ 1 , TRUE ~ 0 )) % >%
filter ( lt 1 == 1 ) % >%
select ( - lt 1 ) % >% collect ()
out = list ( cell = cntsTbl $ cell , count = cntsTbl $ cnt ,
xcm = cmsTbl $ xcm , ycm = cmsTbl $ ycm , xbins = size ,
shape = shape , xbnds = rx , ybnds = ry ,
dimen = bnd , n = n , ncells = nrow ( cntsTbl ) ,
xlab = xName , ylab = yName )
xy = svydbhcell 2 xy ( out )
out = c ( xy , out )
return ( out )
}
svydbhexbin = function ( formula , design , xbins = 3 0 ,
shape = 1 ) {
dsn = design $ clone ()
dsn $ setx ( formula )
dsn $ storename ( " y " , all . vars ( formula )[ 1 ])
dsn $ storename ( " x " , all . vars ( formula )[ - 1 ])
d = dsn $ data
d = d % >% rename ( x = !! sym ( dsn $ names $ x )) % >%
rename ( y = !! sym ( dsn $ names $ y )) % >%
rename ( wt = !! sym ( dsn $ wt )) % >%
filter (! is . na ( x )) % >% filter (! is . na ( y ))
d = compute ( d )
n = d % >% db _ nrow ()
x = d % >% select ( x )
y = d % >% select ( y )
xbnds = c ( dbmin (d , x ) , dbmax (d , x ))
ybnds = c ( dbmin (d , y ) , dbmax (d , y ))
jmax
c1 =
imax
lmax

=
2
=
=

floor ( xbins + 1 . 5 0 0 1 )
* floor (( xbins * shape )/ sqrt ( 3 ) + 1 . 5 0 0 1 )
trunc (( jmax * c 1 - 1 )/ jmax + 1 )
jmax * imax

ans = svydbhbin ( xy = d , x = x , y = y ,
xName = dsn $ names $x , yName = dsn $ names $y ,
cell = as . integer ( lmax ) , cnt = as . integer ( lmax ) ,

68

Appendix A. Codes
xcm = as . integer ( lmax ) , ycm = as . integer ( lmax ) ,
size = xbins , shape = shape ,
rx = as . double ( xbnds ) , ry = as . double ( ybnds ) ,
bnd = as . integer ( c ( imax , jmax )) , n = n )
return ( ans )

}
svydbhexplot = function (d , xlab = d $ xlab , ylab = d $ ylab ) {
pdata = tibble ( x = d $x , y = d $y , count = d $ count ,
xcm = d $ xcm , ycm = d $ ycm )
p = ggplot ( pdata ) + geom _ hex ( aes ( x = x , y = y ,
fill = count ) ,
color = " black " , stat = " identity " ) +
labs ( x = xlab , y = ylab ) +
scale _ fill _ continuous ( trans = " reverse " )
print ( p )
invisible ( p )
}

A.14

svydbcoplot

svydbcoplot = function ( formula , by , design ) {
if (! is _ formula ( by )) {
stop ( " by must be a formula " )
}
y = all . vars ( formula )[ 1 ]
x = all . vars ( formula )[ - 1 ]
dsn = design $ clone ()
by _ var = all . vars ( by )
by = dsn $ data % >% select (!!! syms ( by _ var )) % >%
distinct () % >% arrange (!! sym ( by _ var [ 1 ])) % >%
collect ()
by = split ( by , seq ( nrow ( by )))
filterData = function ( by , dsn , x , y ) {
dsn = dsn $ subset ( paste ( colnames ( by ) , " == " ,
by , collapse = " & " ) , logical = F )
hb = svydbhexbin ( formula , design = dsn )
if ( length ( hb $ x ) | length ( hb $ y )
| length ( hb $ count ) != 0 ) {
cbind ( tibble ( x = hb $x , y =
hb $y , count = hb $ count ) , by )
}
}

A.15. Other Functions
p = lapply ( by , filterData , dsn = dsn ) % >%
Reduce ( rbind , .)
p = ggplot ( p ) + geom _ hex ( aes ( x = x , y = y ,
fill = count ) ,
color = " black " , stat = " identity " ) +
labs ( x = x , y = y ) +
scale _ fill _ continuous ( trans = " reverse " ) +
facet _ wrap ( by _ var , labeller = " label _ both " )
print ( p )
invisible ( p )
}

A.15

Other Functions

dbmin = function ( data , var , asNum = T ) {
var = enquo ( var )
data = data % >% ungroup () % >% select (!! var ) % >%
summarise ( min = min (!! var ))
if ( asNum == T ) {
data % >% pull
} else {
data
}
}
dbmax = function ( data , var , asNum = T ) {
var = enquo ( var )
data = data % >% ungroup () % >% select (!! var ) % >%
summarise ( max = max (!! var ))
if ( asNum == T ) {
data % >% pull
} else {
data
}
}
db _ nrow = function ( data ) {
data % >% ungroup () % >% count () % >% pull ()
}
db _ dim = function ( data ) {
n _ rows = data % >% ungroup () % >% count () % >%
pull ()
n _ cols = ncol ( data )
out = c ( n _ rows , n _ cols )
names ( out ) = c ( " rows " , " cols " )
out

69

70

Appendix A. Codes

}
db _ view = function ( data , num = 1 ) {
if ( class ( data ) == " list " ) {
View ( data [[ num ]] % >% tbl _ df ())
} else {
View ( data % >% tbl _ df )
}
}
db _ rowSums _ mut = function ( data , vars = NULL ,
newRowName = " rSum " ) {
if ( is . null ( vars )) {
s = paste ( colnames ( data ) , collapse = " + " )
} else {
s = paste ( vars , collapse = " + " )
}
q = quote ( mutate ( data , rSum = s ))
eval ( parse ( text = sub ( " rSum " , newRowName ,
sub ( " s " , s , deparse ( q )))))
}
db _ rowSums = function ( data ) {
cn = colnames ( data )
rs = paste ( " ‘" , cn , " ‘" , sep = " " , collapse = " + " )
data % >% transmute ( rowsum = !! parse _ expr ( rs ))
}
db _ cbind = function (x , y ) {
x = x % >% mutate ( ‘___ i ‘ = " key " )
y = y % >% mutate ( ‘___ i ‘ = " key " )
out = inner _ join (x , y , by = " ___ i " , copy = T ) % >%
select ( - ‘___ i ‘)
return ( out )
}
db _ slice = function ( data , n ) {
n = enquo ( n )
data % >% mutate ( ‘___ i ‘ = row _ number ()) % >%
filter ( ‘___ i ‘ <= !! n ) % >% select ( - ‘___ i ‘)
}
dummy _ mut = function ( data , by , withBase = T ,
return . level = F ) {
by = enquo ( by )
dum = data % >% distinct (!! by ) % >% arrange (!! by ) % >%
data . frame () % >% na . omit ()
level = as . character ( dum [ , 1 ])
cs = contrasts ( as . factor ( dum [ , 1 ]))
by _ name = colnames ( dum )

A.15. Other Functions

71

if ( withBase == T ) {
c 1 = c ( 1 , rep ( 0 , nrow ( dum ) - 1 ))
dum = cbind ( dum , c 1 , contrasts ( as . factor ( dum [ ,
1 ])))
name = paste ( colnames ( dum )[ 1 ] , as . character ( dum [ ,
1 ]) , sep = " _ " )
colnames ( dum ) = c ( colnames ( dum )[ 1 ] , name )
} else {
dum = cbind ( dum , contrasts ( as . factor ( dum [ ,
1 ])))
name = paste ( colnames ( dum )[ 1 ] , as . character ( dum [ ,
1 ])[ - 1 ] , sep = " _ " )
colnames ( dum ) = c ( colnames ( dum )[ 1 ] , name )
}
dum = inner _ join ( data , dum , by = by _ name ,
copy = T )
if ( withBase == F ) {
dum = dum % >% select ( -!! by )
}
if ( return . level == T ) {
dum = list ( dum = dum , levels = level )
return ( dum )
} else {
return ( dum )
}
}
db _ cut = function ( var , breaks , data ) {
var = enquo ( var )
var _ name = as . character ( var )[ 2 ]
data = data % >% rename ( vars = !! var ) % >%
filter (! is . na ( vars ))
breaks = breaks [ - 1 ]
trues = seq ( length ( breaks ))
temp _ exprs = paste ( " ifelse ( vars " , " <= " , breaks ,
" ," , trues , " , _ f _) " )
temp _ exprs [ length ( breaks )] = gsub ( pattern = " [_] f [_] " ,
replacement = " NA " , x = temp _ exprs [ length ( breaks )])
mut _ exprs = temp _ exprs [ 1 ]
for ( i in 2 : length ( breaks )) {
mut _ exprs = gsub ( pattern = " [_] f [_] " ,
replacement = temp _ exprs [ i ] , x = mut _ exprs )
}
mut _ exprs = parse _ expr ( mut _ exprs )
data = data % >% mutate ( ‘_ cuts _ ‘ = !! mut _ exprs ) % >%

72

Appendix A. Codes
rename ( ‘:= ‘(!! sym ( var _ name ) , vars )) % >%
rename ( cuts = ‘_ cuts _ ‘)
data

}
db _ cut 2 = function ( var , breaks , right = TRUE ,
data ) {
var = enquo ( var )
mult = diff ( breaks )[ 1 ]
if ( right == TRUE ) {
data = data % >% mutate ( cut = ((!! quo ( mult )) *
ceiling ((!! var )/(!! quo ( mult )))) (!! quo ( mult )))
} else {
data = data % >% mutate ( cut = ((!! quo ( mult )) *
floor ((!! var )/(!! quo ( mult )))))
}
return ( data )
}
svydbVar 2 = function (x , xleft = 1 , xright = 2 ,
st , m _h , data ) {
xleft = x [ xleft ]
xright = x [ xright ]
m = data
m = m % >% select ( xleft = !! sym ( xleft ) ,
xright = !! sym ( xright ) , st = st , m _ h = m _ h )
m _ h = m % >% select ( st , m _ h ) % >% mutate ( m _ h = 1 ) % >%
group _ by ( st ) % >% summarise ( m _ h = sum ( m _ h ))
m = m % >% select ( - m _ h )
m = m % >% mutate ( ztz = xleft * xright ) % >%
group _ by ( st ) % >% summarise ( ztz = sum ( ztz ))
m = left _ join (m , m _h , by = " st " )
m = m % >% mutate ( scaled = ztz * ( m _ h /( m _ h - 1 ))) % >%
select ( scaled ) % >% summarise ( sum ( scaled )) % >%
compute ( temporary = T ) % >% pull ()
return ( m )
}
svydbVar = function (x , st , m _h , data ) {
m = data
m = m % >% select ( x = x , st = st , m _ h = m _ h )
m _ h = m % >% select ( st , m _ h ) % >% mutate ( m _ h = 1 ) % >%
group _ by ( st ) % >% summarise ( m _ h = sum ( m _ h ))
m = m % >% select ( - m _ h )
m = m % >% mutate ( ztz = x * x ) % >% group _ by ( st ) % >%
summarise ( ztz = sum ( ztz ))
m = left _ join (m , m _h , by = " st " )
m = m % >% mutate ( scaled = ztz * ( m _ h /( m _ h - 1 ))) % >%
select ( scaled ) % >% summarise ( sum ( scaled )) % >%

A.15. Other Functions
compute ( temporary = T ) % >% pull ()
return ( m )
}
c 2 f = function ( x ) {
as . formula ( paste ( " ~ " , x , sep = " " ))
}
svydb _ monet _ sampleN = function ( data , n ) {
q = paste ( " SELECT * FROM " , data $ ops $x , " SAMPLE " ,
n)
dbGetQuery ( data $ src $ con , q )
}
as . data . frame . svydbstat = function ( x ) {
ans = cbind ( coef ( x ) , SE ( x ))
colnames ( ans ) = c ( attr (x , " statistic " ) , " SE " )
ans
}
print . svydbstat = function ( xx , ...) {
v <- attr ( xx , " var " )
m = cbind ( xx , sqrt ( v ))
colnames ( m ) = c ( attr ( xx , " statistic " ) , " SE " )
printCoefmat ( m )
}
coef . svydbstat = function ( object , ...) {
attr ( object , " statistic " ) = NULL
attr ( object , " name " ) = NULL
attr ( object , " var " ) = NULL
unclass ( object ) % >% t () % >% as . vector ()
}
SE . svydbstat = function (x , ...) {
s = attr (x , " var " ) % >% sqrt ()
names ( s ) = attr (x , " name " )
return ( s )
}
print . svydbrepstat = function ( xx , ...) {
if ( is . list ( xx )) {
xx = xx $ svydbrepstat
}
v <- attr ( xx , " var " )
m = cbind ( xx , sqrt ( v ))
colnames ( m ) = c ( attr ( xx , " statistic " ) , " SE " )
rownames ( m ) = attr ( xx , " name " )
printCoefmat ( m )
}

73

74

Appendix A. Codes

coef . svydbrepstat = function ( object , ...) {
if ( is . list ( object )) {
object = object $ svydbrepstat
}
attr ( object , " statistic " ) = NULL
attr ( object , " name " ) = NULL
attr ( object , " var " ) = NULL
unclass ( object ) % >% t () % >% as . vector ()
}
SE . svydbrepstat = function (x , ...) {
if ( is . list ( x )) {
x = x $ svydbrepstat
}
s = attr (x , " var " ) % >% sqrt ()
names ( s ) = attr (x , " name " )
return ( s )
}
print . svydblm = function (x , digits =
max ( 3L , getOption ( " digits " ) - 3 L ) , ...) {
print ( x $ design )
cat ( " \ nSurvey design :\ n " )
print ( x $ design $ call )
cat ( " \ nCall :\ n " , paste ( deparse ( x $ call ) , sep = " \ n " ,
collapse = " \ n " ) , " \ n \ n " , sep = " " )
if ( length ( coef ( x ))) {
cat ( " Coefficients :\ n " )
print . default ( format ( coef ( x ) , digits = digits ) ,
print . gap = 2L , quote = FALSE )
} else cat ( " No coefficients \ n " )
cat ( " \ n " )
invisible ( x )
}
summary . svydblm = function ( object ) {
df . r = object $ df . residual
coef . p = coef ( object )
covmat = vcov ( object )
dimnames ( covmat ) = list ( colnames ( coef . p ) ,
colnames ( coef . p ))
var . cf = diag ( covmat )
s . err = sqrt ( var . cf )
tvalue = coef . p / s . err
dn = c ( " Estimate " , " Std . Error " )
pvalue <- 2 * pt ( - abs ( tvalue ) , df . r )
coef . table <- rbind ( coef .p , t ( as . matrix ( s . err )) ,
tvalue , pvalue ) % >% t ()
dimnames ( coef . table ) <- list ( colnames ( coef . p ) ,
c ( dn , " t value " , " Pr ( >| t |) " ))
ans = list ( df . residual = df .r ,
coefficients = coef . table ,

A.15. Other Functions

75

cov . unscaled = covmat , cov . scaled = covmat ,
call = object $ call , design = object $ design )
class ( ans ) <- c ( " summary . svydblm " , " summary . glm " )
return ( ans )
}
print . summary . svydblm = function (x , digits = max ( 3 ,
getOption ( " digits " ) - 3 ) ,
signif . stars = getOption ( " show . signif . stars " ) , ...) {
cat ( " \ nCall :\ n " )
cat ( paste ( deparse ( x $ call ) , sep = " \ n " , collapse = " \ n " ) ,
" \ n \ n " , sep = " " )
cat ( " Survey design :\ n " )
print ( x $ design $ call )
cat ( " \ nCoefficients :\ n " )
coefs <- x $ coefficients
if (! is . null ( aliased <- is . na ( x $ coefficients [ ,
1 ])) && any ( aliased )) {
cn <- names ( aliased )
coefs <- matrix ( NA , length ( aliased ) , 4 ,
dimnames = list ( cn , colnames ( coefs )))
coefs [! aliased , ] <- x $ coefficients
}
printCoefmat ( coefs , digits = digits ,
signif . stars = signif . stars ,
na . print = " NA " , ...)
invisible ( x )
}
predict . svydblm = function ( object , newdata = NULL ,
...) {
tt = delete . response ( object $ terms )
mf = model . frame ( tt , data = newdata ,
xlev = object $ design $ levels )
mm = model . matrix ( tt , mf )
eta = drop ( mm %*% as . vector ( coef ( object )))
attr ( eta , " var " ) = drop ( rowSums (( mm %*% vcov ( object )) *
mm ))
attr ( eta , " statistic " ) = " link "
class ( eta ) <- " svydbstat "
eta
}
vcov . svydblm = function (x , ...) {
x $ cov . unscaled
}

77

Appendix B

Result Tables
All units are in seconds.

B.1

Total

B.1.1

Numeric variable

1
2
3
4
5
6
7
8
9
10
11
12

B.1.2

Observations
250000
500000
750000
1000000
1250000
1500000
1750000
2000000
2500000
3000000
3500000
4000000

Survey
0.43
0.88
1.38
1.76
2.35
2.80
3.26
3.95
-

svydb.local
0.06
0.08
0.11
0.15
0.17
0.21
0.28
0.26
-

svydb.database
0.40
0.41
0.61
0.72
0.81
0.94
0.97
1.15
1.17
1.56
1.61
1.92

Categorical variable with 7 levels

1
2
3
4
5
6
7
8
9
10
11
12

Observations
250000
500000
750000
1000000
1250000
1500000
1750000
2000000
2500000
3000000
3500000
4000000

Survey
0.73
1.25
1.87
2.58
3.10
3.94
4.42
5.27
-

svydb.local
0.22
0.28
0.34
0.46
0.51
0.61
0.71
0.79
-

svydb.database
1.20
1.50
1.76
2.03
2.28
2.65
2.92
3.20
3.63
4.25
4.67
5.21

78

Appendix B. Result Tables

B.2

Mean

B.2.1

Numeric variable

1
2
3
4
5
6
7
8
9
10
11
12

B.2.2

Observations
250000
500000
750000
1000000
1250000
1500000
1750000
2000000
2500000
3000000
3500000
4000000

Survey
0.43
0.90
1.35
1.86
2.13
2.77
3.26
3.60
-

svydb.local
0.09
0.11
0.14
0.21
0.23
0.27
0.30
0.34
-

svydb.database
0.63
0.79
0.96
1.24
1.31
1.48
1.62
1.85
2.09
2.39
2.66
3.09

Categorical variable with 7 levels

1
2
3
4
5
6
7
8
9
10
11
12

Observations
250000
500000
750000
1000000
1250000
1500000
1750000
2000000
2500000
3000000
3500000
4000000

Survey
0.72
1.25
1.90
2.49
3.26
4.02
4.85
5.55
-

svydb.local
0.21
0.35
0.47
0.70
0.76
0.89
1.02
0.94
-

svydb.database
1.65
2.49
2.73
3.26
3.83
4.50
5.04
5.63
6.68
7.53
8.70
9.72

B.3. Regression

B.3

Regression

1
2
3
4
5
6
7
8
9
10
11
12

B.4

79

Observations
250000
500000
750000
1000000
1250000
1500000
1750000
2000000
2500000
3000000
3500000
4000000

Survey
1.18
2.00
3.16
4.50
5.65
6.77
8.13
9.75
-

svydb.local
0.68
1.03
1.30
1.31
1.65
1.97
2.24
2.68
-

svydb.database
2.95
4.24
5.31
6.56
7.75
8.70
10.09
10.72
12.31
14.31
16.32
18.22

Survey
0.58
1.16
1.55
2.18
2.80
3.45
4.24
4.82
-

svydb.local
0.08
0.18
0.26
0.29
0.28
0.31
0.27
0.32
-

svydb.database
0.36
0.51
0.63
0.83
0.93
1.04
1.17
1.40
1.49
1.73
1.98
2.25

Quantiles

1
2
3
4
5
6
7
8
9
10
11
12

Observations
250000
500000
750000
1000000
1250000
1500000
1750000
2000000
2500000
3000000
3500000
4000000

80

B.5

Appendix B. Result Tables

Survey Tables

1
2
3
4
5
6
7
8
9
10
11
12

B.6

Survey
0.18
0.39
0.56
0.78
0.97
1.16
1.38
1.56
-

svydb.local
0.26
0.49
0.69
0.76
1.00
1.11
1.37
1.47
-

svydb.database
0.67
1.02
1.35
1.69
2.05
2.44
2.56
2.97
3.45
4.05
4.63
5.20

svydb.local
0.46
0.74
1.13
1.45
-

svydb.database
2.12
3.49
4.81
5.86
7.10
9.34
10.62
14.75

Total - Replicate Weights

1
2
3
4
5
6
7
8

B.7

Observations
250000
500000
750000
1000000
1250000
1500000
1750000
2000000
2500000
3000000
3500000
4000000

Observations
250000
500000
750000
1000000
1250000
1500000
1750000
2000000

Survey
3.54
9.03
12.31
21.57
-

Mean - Replicate Weights

1
2
3
4
5
6
7
8

Observations
250000
500000
750000
1000000
1250000
1500000
1750000
2000000

Survey
3.57
8.69
14.25
20.27
-

svydb.local
0.66
0.98
1.61
1.89
-

svydb.database
3.66
5.68
7.23
8.72
10.36
12.24
13.97
16.02

B.8. Histogram

B.8

Histogram

1
2
3
4
5
6
7
8
9
10
11
12

B.9

81

Observations
250000
500000
750000
1000000
1250000
1500000
1750000
2000000
2500000
3000000
3500000
4000000

Survey
0.97
2.07
3.14
4.03
5.15
6.12
7.50
8.57
-

svydb.local
0.58
0.75
0.90
1.05
1.15
1.42
1.48
1.66
-

svydb.database
1.57
2.56
2.84
3.49
4.12
4.79
5.24
6.05
7.03
7.94
9.15
10.34

Survey
2.84
4.96
7.44
10.36
12.83
14.96
18.22
22.12
-

svydb.local
1.21
1.39
1.54
1.69
1.90
2.01
2.14
2.28
-

svydb.database
4.37
5.20
6.16
7.01
7.88
8.92
9.61
10.31
11.38
13.01
14.56
16.20

Boxplot

1
2
3
4
5
6
7
8
9
10
11
12

Observations
250000
500000
750000
1000000
1250000
1500000
1750000
2000000
2500000
3000000
3500000
4000000

82

B.10

Appendix B. Result Tables

Hexagon Binning

1
2
3
4
5
6
7
8
9
10
11
12

Observations
250000
500000
750000
1000000
1250000
1500000
1750000
2000000
2500000
3000000
3500000
4000000

Survey
0.41
0.56
0.99
1.00
1.50
1.54
1.97
2.32
-

svydb.local
0.92
1.23
1.53
1.97
2.17
2.47
2.61
2.92
-

svydb.database
2.05
3.23
4.27
5.53
6.40
7.33
8.24
9.29
10.65
12.59
14.41
16.40

83

Bibliography
Bache, Stefan Milton and Hadley Wickham (2014). magrittr: A Forward-Pipe Operator
for R. R package version 1.5. URL: https : / / CRAN . R - project . org / package =
magrittr.
Carr, D. B. et al. (1987). “Scatterplot Matrix Techniques for Large N”. In: Journal of the
American Statistical Association 82.398, pp. 424–436. ISSN: 01621459. URL: http :
//www.jstor.org/stable/2289444.
Carr, Dan et al. (2018). hexbin: Hexagonal Binning Routines. R package version 1.27.2.
URL : https://CRAN.R-project.org/package=hexbin.
Chang, Winston (2017). R6: Classes with Reference Semantics. R package version 2.2.2.
URL : https://CRAN.R-project.org/package=R6.
Henry, Lionel and Hadley Wickham (2018). rlang: Functions for Base Types and Core R
and ’Tidyverse’ Features. R package version 0.2.0. URL: https://CRAN.R-project.
org/package=rlang.
Horvitz, D. G. and D. J. Thompson (1952). “A Generalization of Sampling Without
Replacement From a Finite Universe”. In: Journal of the American Statistical Association 47.260, pp. 663–685. ISSN: 01621459. URL: http://www.jstor.org/stable/
2280784.
Lewin-Koh, Nicholas (2016). Hexagon Binning: an Overview. URL: https://cran.rproject.org/web/packages/hexbin/vignettes/hexagon_binning.pdf.
Lumley, Thomas (2004). “Analysis of Complex Survey Samples”. In: Journal of Statistical Software 9.1. R package verson 2.2, pp. 1–19.
— (2014). sqlsurvey: analysis of very large complex survey samples (experimental). R package version 0.6-11/r41. URL: https : / / R - Forge . R - project . org / projects /
sqlsurvey/.
Luraschi, Javier et al. (2018). sparklyr: R Interface to Apache Spark. R package version
0.8.3. URL: https://CRAN.R-project.org/package=sparklyr.
MonetDB-B.V. (2008). MonetDB Database System. URL: https://www.monetdb.org/.
Ruiz, Edgar (2018). dbplot: Simplifies Plotting Data Inside Databases. R package version
0.3.0. URL: https://CRAN.R-project.org/package=dbplot.
Sievert, Carson et al. (2017). plotly: Create Interactive Web Graphics via ’plotly.js’. R
package version 4.7.1. URL: https://CRAN.R-project.org/package=plotly.
Wickham, Hadley (2009). ggplot2: Elegant Graphics for Data Analysis. Springer-Verlag
New York. ISBN: 978-0-387-98140-6. URL: http://ggplot2.org.
Wickham, Hadley, Peter Danenberg, and Manuel Eugster (2017). roxygen2: In-Line
Documentation for R. R package version 6.0.1. URL: https://CRAN.R- project.
org/package=roxygen2.
Wickham, Hadley, Jim Hester, and Winston Chang (2018). devtools: Tools to Make Developing R Packages Easier. R package version 1.13.5. URL: https : / / CRAN . R project.org/package=devtools.
Wickham, Hadley and Edgar Ruiz (2018). dbplyr: A ’dplyr’ Back End for Databases. R
package version 1.2.1. URL: https://CRAN.R-project.org/package=dbplyr.
Wickham, Hadley et al. (2017). dplyr: A Grammar of Data Manipulation. R package
version 0.7.4. URL: https://CRAN.R-project.org/package=dplyr.

84

BIBLIOGRAPHY

Xie, Yihui (2017). formatR: Format R Code Automatically. R package version 1.5. URL:
https://CRAN.R-project.org/package=formatR.

